\chapter{Problemformulering}
Problemformuleringen, som ønskes undersøgt i dette bachelorprojekt, er udarbejdet i samarbejde med den øvrige forskningsgruppe, se Forord.  

\section{Udviklingsprojekt}
I dag bliver fraktureringen af den degenererede biologiske hjerteklap under BF-ViV proceduren blot observeret angiografisk på fluoroskopi optagelser af operatøren. De eneste indikationer, som operatøren får ved fraktureringsøjeblikket er, at operatøren visuelt på fluoroskopi kan se, at ballonformen ændres fra timeglas- til cylinderform samt at trykket falder på det analoge manometer. Fraktureringen understøttes nogle gange af en auditiv 'klik'-lyd.

Formålet med detekteringssystemet er at udvikle et system, der kan støtte og bekræfte operatørens visuelle observation i, at fraktureringen har fundet sted. Systemet skal kunne detektere det transiente trykfald, der forekommer, når ballonen ændrer sig fra timeglas- til cylinderform, grundet ballonens udvidelse til sin fulde diameter. 

Ydermere ønskes det, at detekteringssystemet skal anvendes i forskningsprojektet i dette bachelorprojekt, samt til fremtidige eksperimentelle forsøg, indenfor forskningen af proceduren BF-ViV.     

Følgende overordnede problemstillinger ønskes derfor undersøgt i forbindelse med udvikling af detekteringssystemet:
\begin{itemize}
	\item Undersøge det kliniske behov for et detekteringssystem
	\begin{itemize}
		\item Interview med operatør 
		\item Litteraturstudie 
	\end{itemize}
	\item Identificer en anvendelig og tilgængelig tryktransducer, der kan måle et tryk indenfor det nødvendige måleområde
	\item Udvikle en detekteringsalgoritme
	\item Evaluere og teste den udviklede prototype	
\end{itemize}


\section{Forskningsprojekt}
Tidligere i forskningen er det blevet vist, at fraktureringstrykket i 3D-printede modeller af raske aortarødder er statistisk signifikant forskelligt fra frakturtrykket i 3D-printede modeller af forkalkede aortarødder \cite{rapport}. Betydningen af forskellige kalcificeringsgrader er endnu ikke blevet undersøgt. 

I dette delprojekt er formålet derfor at undersøge betydningen for det nødvendige fraktureringstryk i 3D-printede aortarødder med varierende kalcificeringsgrad. Til udførsel af dette studie vil det udviklede detekteringssystem blive anvendt. 

Følgende overordnede problemstillinger ønskes derfor undersøgt i forbindelse med identificering af det nødvendige fraktureringstryk ved forskellige kalcificeringsgrader:
\begin{itemize} 
	\item Undersøg mængden af kalk ved patientspecifikke CT scanninger
	\item Producer 3D-printede aortamodeller med forskellige kalcificeringsgrader
	\item Undesøge hvorvidt de forskellige kalcificeringsgrader påvirker fraktureringstrykket 
\end{itemize}







