\chapter{Theory}


\section{What is telemedicine and ICT}

Considerations of Telemedicine in the delivery of modern healthcare



The first use of Telemedicine was in 1877. A group of doctors made a communication network towards the drug store by using the telephones. The first video consultation between a doctor and a patient took place in 1927. In the 1950s a two way television group therapy took place in Alaska. In the 1970s NASA built \textit{Space Technology Applied to Rural Papago Advanced Health Care (STARPAHC)}. This system was able to communicate with a two-way radio, audio and data. It wasn't until the eighties where the technology had renewed interest, due to high cost, lack of suitable technologies and unacceptance. At this point the military picked up the idea of the usage of telemedicine in combat. The use of the technology in the military has extended to hospitals throughout the world. 


Telemedicine is a generic term that covers different types of healthcare which is provided digitally and in distance. Telemedicine range from teleconsultations to telesurgery. Telemedicine has made it possible to give specialized care and diagnostic medicine for people in rural and remote areas. The introduction of telemedicine has changed the traditional doctor-patient relationship. 
ICT is a information and communication technology which allow people to interact in the digital world. Telemedicine use this technology as there digital communaction method. ICT has drastically changed the way the world in general communicates, work, learn and live.   
The usage of ICT in telemedicine have made cost-effective treatment options available due to reduced traveling expenses, decreasing hospital readmission rates, and maximization of consultations. Though providing medical care with the usage of telemedicine opens important medical, ethical and legal issues that must be addressed\cite{considerations}.

% Ovenstående skal vi kigge ind i, etik og sådan!!!!!!!!
% Måske et begreb vi kan bruge "store and forward"
