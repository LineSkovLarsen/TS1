\chapter{Empirical data}
\label{empirical}

The empirical data is collected from interviews both with a perspective from Herning Hospital, Herning Municipality and the involved patients. The outcome of the interviews is presented in the below sections.  

\section{Interview with Vibeke Lynggard, post.doc, Cardiovascular Clinic, Herning Hospital}
The interview was held as a semi structured interview and was a conversation based two-way communication. The questions was premade, but the interview was open for clarifying questions. The interview was audio-recorded and afterword the record was transcribed, see Appendix XX for the transcription.    

Vibeke Lynggaard is working as a project nurse at the clinic for cardiovascular research. The interview was also held at the cardiovascular clinic. The interview was conducted in order to gain information on telemedicine rehabilitation from a healthcare professionals point of view. Through this interview it was possible to obtain information on how telemedicine rehabilitation can be used to support the already used rehabilitation process. By the interview it was also possible to obtain an opinion on which limitations and disadvantages a telemedicine solution gives. By the information based on this interview it was possible to give an idea on how telemedicine rehabilitation can be used in the Danish Healthcare System. 

\subsection{Outcome of the interview} 
Cardiac rehabilitation has been a subject to the municipalities since the 1 of January 2017. Before that it was the regions responsibility to offer rehabilitation for cardiac patients. Only one team of 12 patients are directly connected to the hospital. These patients are severely ill and therefore in need to be in contact with the hospital throughout their rehabilitation progress. Cardiac rehabilitation consists of exercise, education and social aspects and it is important that all three collaborates. Both at the hospital and in the municipalities the patients do cardiac rehabilitation 2 times a week in 12 weeks. Furthermore, they are being educated about their illness and healthy lifestyle habits once a week. These 12 weeks of rehabilitation is described as phase 2 rehabilitation. Afterword patients are offered phase 3 rehabilitation. At the beginning of phase 2 rehabilitation the patients makes a maximal symptom-based stress test. This gives a starting point for the rehabilitation progress. Furthermore, they measure weight, height, waistline, blood pressure and heart rate. After 12 weeks they will do the test and measurements again and hereby the healthcare professionals are able to see how the patients has developed through the progress. The physiotherapist coaches the patients through a program where both cardio stress and muscle stretching are included. There are some national guidelines which indicates how much intensity and how many reputations the patients are able to do. It is necessary to keep in mind that cardiac patients struggles with different illnesses, such as reduced pump function, those who have had a new heart valve and those who has got a pacemaker. Therefore, cardiac patients got different needs and have to do different exercises.  

\textbf{How could telemedicine be included in rehabilitation?} \newline 
Phase 2 rehabilitation is mandatory to offer cardiac patients. This rehabilitation program is a class 1 recommendation from the American and European heart institutes and there is evidence that it is working. Therefore, it would be a hard process to replace phase 2 rehabilitation with telemedicine rehabilitation. All patients are being offered phase 3 rehabilitation, but most patients refuses to undergo this program. Mostly they refuse phase 3 as they want to get back to the labor market and a normal life. Unfortunately, most patients stops with common care and does not continue their healthy lifestyle after the rehabilitation program. Therefore, it would be highly relevant to introduce a telemedicine solution after phase 2 rehabilitation. Furthermore, phase 3 rehabilitation is about continuing and maintaining decent exercises and dietary habits and for example continue a smoking cessation. 

More ever patients are being screened for depression at the beginning of their illness. They complete the questionnaire HADS – The Hospital Anxiety and Depression Scale. There is an algorithm behind the scale which indicates if the patient need to contact the doctor. It could be beneficial to perform a depression score once a month and hereby catch patients who is in risk of getting a depression. This questionnaire could also be a part of phase 3 and could easily be included in a telemedicine rehabilitation program. 

\textbf{Is it possible to do all measurements at home?} \newline 
Waistline can be hard to do by the patients themselves. Usually healthcare professionals do this measurement to make sure there is no bias. The objectivity disappears if the patient has to do that kind of personal measurements themselves. Furthermore, Vibeke Lynggaard mentioned that it could be relevant to connect a censor to the telemedicine platform. This could for example be used as a reminder to take medicine. Hereby the patient need to interact with the platform to let the system know that they have taken the medicine, otherwise an alarm will start. Another score that is used today is a score to indicate how nicotine dependent the patient is. The score will let the patient know whether they need to use a nicotine plaster or chewing gum. This could easily take place at the patients home as well. 

\textbf{How could telemedicine be used in rehabilitation?} \newline 
It would be beneficial to track how the patients complies with healthy lifestyle and behavior. Moreover, if the patient could interact with healthcare professional through the platform it would be very useful. By this communication platform it would be possible for the patient to ask questions about the illness and how to maintain a healthy lifestyle. For healthcare professionals it would be possible to keep an eye on the patient and keep track on their behavior. 

The rehabilitation program that is used today includes former patients to communicate experienced knowledge. This could also be included in telemedicine rehabilitation where former patients could be connected the platform and hereby it is possible for former patients and current patients to be in contact when needed.   

Furthermore, at the current rehabilitation program patients are offered a program at a dietitian. Usually the program is within two days and concludes a presentation and some practical exercises. Not all patients accepts this offer and therefor it would be preferable to include meal plans, diet diaries or television chefs at the platform. By this platform the patient does not need to go to the dietitian, but they will be able to obtain the information through the platform. This would possible seem easier and hopefully more patients would continue a healthy lifestyle. 

By introducing telemedicine in rehabilitation both the patient and healthcare professionals would experience time savings as they does not have to spent time on transportation. If telemedicine could be used to keep a healthy lifestyle for more patients, and hereby reduce readmission, a health economic reduce will also be a positive outcome.

\textbf{Participants in rehabilitation program in Regional Hospital West Jutland} \newline
Vibeke Lynggard has taken part in a trial which where carried out in collaboration with Regional Hospital West Jutland. In the recruitment period of the trial 1642 cardiac patients where hospitalized and only 50\% participated in the rehabilitation program. The reason that only 50\% participated was that some patients did not meet the inclusion criteria, some where not referred to cardiac rehabilitation, some declined to participate and some were rehabilitated in Primary Health Care. Furthermore, the trial showed that 20\% did not complete the rehabilitation program \cite{rehabiliteringDK}. By this trial it can be concluded that a very large group of patients do not undergo the rehabilitation program that is being offered. Hopefully by introducing telemedicine, and herby a more technological and easy access rehabilitation program, this large number of patients could be reduced. 

\textbf{Limitations and disadvantages} \newline
The social aspect with other patients can be hard to obtain by using telemedicine instead of cardiac classes. To most patients it gives motivation if the teamwork and dynamic between the patients is good. There has been made some qualitative research interviews which indicates that the social aspect is highly important. Furthermore, there are some ethical challenges that has to be taken into account. For example, patients must be offered the best-known rehabilitation program and as there is highly evidence that the rehabilitation program that is used today is the best this one must be the one that is being offered.  

\section{Interview with Hanne Voldgaard Nielsen, Professional leader, Health, Training and Assistive Technology, Herning Municipality and Eva Klose Jensen, Leader, Rehabilitation and Health, Herning Municipality}\label{sec: evahanne}

The interview was held as a semi structured interview and was conversations based between the interviewer and the two interviewees. The interview was held to give an perspective on how rehabilitation is working in Herning Municipality and to achieve information on cost-effectiveness data. The questions was premade and was based on previous and comparable cost effectiveness analysis from other similar studies \cite{costeffect, usingeffect}. Notes were taken during the interview and can be seen in Appendix XXX.

Hanne Voldgaard Nielsen and Eva Klose Jensen are both working with cardiac rehabilitation in Herning Municipality. The interview was held at Holing Sport Center in Herning which is normally used for rehabilitation. The main purpose for this interview was to obtain cost-effectiveness data. By the interview it was almost possible to achieve all the cost data that was sought. The only data that was not possible to get was the number of readmissions. Unfortunately, is was not possible to get any effectiveness information as the rehabilitation team does not use and calculate on that specific data. Furthermore, the interviewees gave a subjective opinion on how cardiac rehabilitation is working in Herning Municipality.              

\subsection{Outcome of the interview} 
In Herning Municipality 200 patients are being offered the rehabilitation program. \Cref{tab: reg} shows patients who did not want to participate on cardiac rehabilitation. As seen on the table the percentage of participants is very high, and only five patients turned down the offer. 

\begin{table}[H]
\begin{longtabu} to \linewidth{@{}l c c X[j]@{}}
    \textbf{Date} & \textbf{Ends before test} & \textbf{Ends after test} & \textbf{Reason} \\[-1ex]
    \midrule
     17.01.18   &    X &  & Fails to appear \\ \hline
     06.02.18   &    X &  & Dementia – tried to start the program. Does wish to continue \\ \hline
     18.04.18   &    X &  & Regretted that he said yes \\ \hline
     30.04.18   &    X &  & Called for start test - announces cancellation. Ends via non-disclosure letter \\ \hline
     23.04.18   &      & X & Depression \\ \hline
    \newline
   \end{longtabu}
\caption{Registration cardiac rehabilitation 2018}
\label{tab: reg}
\end{table}

In Herning Municipality, the rehabilitation program is offered in Sport Center Holing which is very centrally located for patients in Herning Municipality. It is known from other Municipalities that patients who lives in rural areas seem to turn down the offer as the rehabilitation program is being offered to far away from their homes. 

In Herning Municipality, the rehabilitation department agreed on marking the best possible solution where the desire was to achieve the best possible result for cardiac patients. Therefore, the solution is not the cheapest as the main focus is not on the financial part. The primary focus is on the patient and how to get them through a rehabilitation process in the best possible way with the best possible outcome. The nurses used within the rehabilitation program is hired from the cardiac department at the hospital. It is important that the nurses have the right competences to work with cardiac patients and therefore they are more experienced and on a higher salary step. The same applies for the physiotherapists. When the rehabilitation department became a subject to the municipalities the rehabilitation department had to find the right place to perform and offer the rehabilitation program. I Herning they did not have access to a health center and therefore they were required to invest all equipment necessary to open op a rehabilitation center. Therefore, it is important to keep in mind that the prices seen in \Cref{tab: MC} and \Cref{tab: MeC} are implementation rates and hereby a lump sum payment.   

Beside nurses and physiotherapists, a dietitian is available through the rehabilitation process. The dietitian participates four times a year for three hours. Furthermore, cardiac patients are being offered smoking cessation during and after the rehabilitation program. Most patients do not have the time and energy to undergo a smoking cessation while they participate in the rehabilitation program. Therefore, it is possible to do a smoking cessation after the 12 weeks of rehabilitation and it will still be paid by the municipality.   

An important part in the rehabilitation program is that a previous patient participates in the program. By this concept patients are able to share experiences and to get knowledge on how to deal with a chronical disease. This is a very important factor to many of the patients as it is very trustworthy to hear personal disease and health experiences. 

By the interview it was mentioned that patients are being pressed hard physically to obtain the best result. A few times it has been experienced that some patients feel ill and are in need for personal help and assistance. As this patient group suffer from heart diseases it is important to keep in mind that they are in risk of getting a heart attack while they are doing their exercises. This is an important information to be aware of if telemedicine is introduced in rehabilitation. If patients are doing their exercise via telemedicine at home instead of participating at a team at the rehabilitation center, they are not able to receive the same urgent help if a heart attack or another emergency health situation appears.         

\section{Economy and effect}

The economy aspects of this study is conducted in collaboration with Herning Municipality. The primary data for the economic analysis was gathered at Sportcenter Herning, which is the location of cardiac rehabilitation in the municipality. The data was collected as an open semi-structured interview with Eva Klose Jensen, Rehabilitation and health promotion manager in Herning Municipality and Hanne Volsgaard Nielsen, Professional leader at Herning Municipality. The data from this interview will be used for the control group resources of the standard CR treatment. The outcome for this interview can be seen in \Cref{sec: evahanne}. Hanne and Eva were not aware of the expenses for the database HjerteKomMidt. To collect the cost of the database they referred to the region whereas the cost of the database could be enlightened \cite{hjerteKomMidt}.

The economy for the intervention group is based on estimations because a trial is not conducted in this study. vCare for cardiac patients is not yet developed and there is not any guidelines to what elements an ICT solution should include for CR. Due to the lack of rules in this area the project group made three possible packages including professions, materials, medical equipment, IT, and other costs. The three packages is produced based on knowledge gained from the meeting with Herning Municipality, providers of similar products, former studies and estimations \cite{sofoklis, costeffect, effects, usingeffect}. Out of the three produced packages one is chosen for comparison between treatment of the control- and intervention group.

The packages are divided in bronze, silver and gold. With all packages the patient will be provided with a tablet and a license for the program and proper interaction with health professionals. 

The Bronze package is the minimum possible set up, where the patient will carry out the training in a fitness center such as Fitness World. In this way no training equipment will be needed in the patient’s home. 

The silver packages includes training equipment to achieve home training otherwise the set-up of the bronze and silver package is somewhat equal. 

In the gold package sensors are included to monitor the patients during training and to keep an eye on their state of health. Furthermore, the combination of professions needed for this task is slightly different. The solution is a store and forward solution where data will be collected and afterwards delivered to healthcare professionals.  

To compare the control group with the intervention group in the cost-effectiveness analysis the silver package is the most suitable package hence the silver package is the most comparable to the equipment used in the control group. The bronze and gold package will not be further described in the report, but the economic setups are included in Appendix XXXX.

This case study is not a trial so there is need of data from a former conducted trial to look closer into the effect of implementing a ICT solution in rehabilitation of cardiac patients. To do so, a study from 2015 in Belgium was chosen \cite{costeffect}. The set up for the trial is comparable to the danish setup, see \Cref{belgium}. It is a randomized controlled trial which is looking into the cost-effectiveness of tele rehabilitation with the ICER method. The study uses the same methodology and set up as wanted in this paper but with focus on the Belgian Healthcare System. In this paper they measure effect in Quality Adjusted Life Years. The result of the papers QALY is comparable to this project as Belgium and Denmark are following the same guidelines provided by the European Union for cardiac patients. To collect the QALY the EQ-5D questionnaire was used before, after and under the trial. Therefore, it was chosen to use the collected QALY of the intervention group from this paper. Herning Municipality do not collect effectiveness data on their patients and due to the lack of data for the control group the QALY of the control group is used from the Belgian project \cite{costeffect}.

\section{Open discussion with cardiac patients, Sport Center Holing, Herning}
\label{patientinterview} 

The interview took place at Sport Center Holing and was conducted both as a micro ethnography and an open group discussion. 12 patients participated, and the interview was held as a part of the rehabilitation program that the patients are following. The interviewer participated in the physical rehabilitation program followed by the education program within knowledge on risk factors to develop heart disease. During both the physical rehabilitation program and the education program, field notes were taken and got written down after the session. After the education program the interviewer had time to ask questions. The questions was premade and were focusing on the subjective perspective to how telemedicine could be used in rehabilitation. The interview was audio-recorded, and afterword transcribed, see Appendix XX for the transcription. 

The patients were mostly elderly retired citizens who suffer from heart disease, only a few were still on the labor market. One of the patients have not had a heart disease but were in higher risk of becoming of cardiac patient. Therefore, the rehabilitation program in Herning Municipality is both offered to cardiac patients but also to citizens in high risk of getting a heart disease. By offering the rehabilitation program to citizens in high risk of getting a heart disease, these citizens could possibly get the right help and information about heart disease and hereby avoid becoming a cardiac patient. By this open group discussion, it was possible to obtain opinions on the use of telemedicine rehabilitation by the patients themselves. It was also possible to get an impression on how the rehabilitation program works in Herning Municipality and how the patients find the program. 

\textbf{Outcome of the interview}

To this group of patients, they could not imagine home based rehabilitation as a replacement to the 12-week rehabilitation program that they follow at the sport center. They do not think they would be able to the exercise at home as they had to do it by themselves instead of doing it together on a team. To them the social aspect is very important. A patient states, “We have a great relationship to each other. It means a lot that it is not just workout but also social. We meet in good time before we start and speak a little afterwards as well: ”To do rehabilitation by themselves is not as motivating as being on a team. It is too impersonal and to talk with other patients will be missed. 

Another important factor is the safety by participating at the sport center compared to home rehabilitation. The patients feel safe at the center as there is a nurse and a physiotherapist while they are doing the exercises. That would not be the same if rehabilitation took place at the patient’s home. If a patient feel ill during home rehabilitation it would take some time before they get the right help. The patients are being pushed very hard during training and sometimes patients feel a bit dizzy after an exercise. 

Telemedicine rehabilitation could possibly be used after the 12 weeks of center-based rehabilitation. A patient mention: “That would definitely be a motivation to continue”. The patients could see an idea in developing an app where different training programs and nutritional advices were regularly uploaded. It was also discussed how measurements, e.g. blood pressure could be measured at home and sent directly to doctors at the hospital. To these patients that was seen as a positive monitoring and would give a sense of security.  

Overall this patient group would prefer to keep the 12 weeks of center-based rehabilitation. At the center they learn what exercise to do and they learn to do them the right way. An important information was mentioned by one of the nurses who pointed out that unfortunately after three years most patients are back at the same state of health as they were before they started rehabilitation. Therefore, to do some kind of rehabilitation program after the 12 weeks could possibly have an positive effect. One patient mentioned that it would be beneficial to have a progress after the 12 weeks of center-based rehabilitation. “It would be helpful if it could be tracked that the citizen followed the training program. Of course, it is beneficial to my own health, but it gives motivation that someone is looking after me”. Even though, patients learn training exercises and get healthy nutrition advises it can be hard to stay motivated and continue the new healthy lifestyle. Therefore, some kind of telemedicine solution after the 12 weeks of center-based rehabilitation is seen as a possible and positive solution, both to patients and healthcare professionals.        


















































