\chapter{Empirical data}
\label{empirical}


The empirical data is collected from interviews both with Vibeke Lynggaard, post.doc at the Cardiovascular Clinic, Herning Hospital and PATIENT?!?!??! The data has been analyzed in section \ref{analysis}. 

%Vi skal indskrive EVA og syfoklis for økonomiske data

\section{Interview with Vibeke Lynggard, post.doc, Cardiovascular Clinic, Herning Hospital}
The interview was held as a semi structured interview and was a conversation based two-way communication. The questions was premade, but the interview was open for clarifying questions. The interview was audio-recorded and afterword the record was transcribed, see Appendix XX for the transcription.    

Vibeke Lynggaard is working as a project nurse at the clinic for cardiovascular research. The interview was also held at the cardiovascular clinic. The interview was conducted in order to gain information on telemedicine rehabilitation from a healthcare professionals point of view. Through this interview it was possible to obtain information on how telemedicine rehabilitation can be used to support the already used rehabilitation process. By the interview it was also possible to obtain an opinion on which limitations and disadvantages a telemedicine solution gives. By the information based on this interview it was possible to give an idea on how telemedicine rehabilitation can be used in the Danish healthcare system. 

\subsection{Outcome of the interview} 
Cardiac rehabilitation has been a subject to the municipalities since the 1 of January 2017. Before that it was the regions responsibility to offer rehabilitation for cardiac patients. Only one team of 12 patients are directly connected to the hospital. These patients are severely ill and therefore in need to be in contact with the hospital throughout their rehabilitation progress. Cardiac rehabilitation consists of exercise, education and social and it is important that all three collaborates. Both at the hospital and in the municipalities the patients do cardiac rehabilitation 2 times a week in 12 weeks. Furthermore, they are being educated about their illness and healthy lifestyle habits once a week. These 12 weeks of rehabilitation is described as phase 2 rehabilitation. Afterword patients are offered phase 3 rehabilitation. At the beginning of phase 2 rehabilitation the patients makes a maximal symptom-based stress test. This gives a starting point for the rehabilitation progress. Furthermore, they measure weight, height, waistline, blood pressure and heart rate. After 12 weeks they will do the test and measurements again and hereby the healthcare professionals are able to see how the patients has developed through the progress. The physiotherapist coaches the patients through a program where both cardio stress and muscle stretching are included. There are some national guidelines which indicates how much intensity and how many reputations the patients are able to do. It is necessary to keep in mind that cardiac patients struggles with different illnesses, such as reduced pump function, those who have had a new heart valve and those who has got a pacemaker. Therefore, cardiac patients got different needs and have to do different exercises.  

\textbf{How could telemedicine be included in rehabilitation?} \newline 
Phase 2 rehabilitation is mandatory to offer cardiac patients. This rehabilitation program is a class 1 recommendation from the American and European heart institutes and there is evidence that it is working. Therefore, it would be a hard process to replace phase 2 rehabilitation with telemedicine rehabilitation. All patients are being offered phase 3 rehabilitation, but most patients refuses to undergo this program. Mostly they refuse phase 3 as they want to get back to the labor market and a normal life. Unfortunately, most patients stops with common care and doesn’t continue their healthy lifestyle after the rehabilitation program. Therefore, it would be highly relevant to introduce a telemedicine solution after phase 2 rehabilitation. Furthermore, phase 3 rehabilitation is about continuing and maintaining decent exercises and dietary habits and for example continue a smoking cessation. 

More ever patients are being screened for depression at the beginning of their illness. They complete the questionnaire HADS – The Hospital Anxiety and Depression Scale. There is an algorithm behind the scale which indicates if the patient need to contact the doctor. It could be beneficial to perform a depression score once a month and hereby catch patients who is in risk of getting a depression. This questionnaire could also be a part of phase 3 and could easily be included in a telemedicine rehabilitation program. 

\textbf{Is it possible to do all measurements at home?} \newline 
Waistline can be hard to do by the patients themselves. Usually healthcare professionals do this measurement to make sure there is no bias. The objectivity disappears if the patient has to do that kind of personal measurements themselves. Furthermore, Vibeke Lynggaard mentioned that it could be relevant to connect a censor to the telemedicine platform. This could for example be used as a reminder to take medicine. Hereby the patient need to interact with the platform to let the system know that they have taken the medicine, otherwise an alarm will start. Another score that is used today is a score to indicate how nicotine dependent the patient is. The score will let the patient know whether they need to use a nicotine plaster or chewing gum. This could easily take place at the patients home as well. 

\textbf{How could telemedicine be used in rehabilitation?} \newline 
It would be beneficial to track how the patients complies with healthy lifestyle and behavior. Moreover, if the patient could interact with healthcare professional through the platform it would be very useful. By this communication platform it would be possible for the patient to ask questions about the illness and how to maintain a healthy lifestyle. For healthcare professionals it would be possible to keep an eye on the patient and keep track on their behavior. 

The rehabilitation program that is used today includes former patients to communicate experienced knowledge. This could also be included in telemedicine rehabilitation where former patients could be connected the platform and hereby it is possible for former patients and current patients to be in contact when needed.   

Furthermore, at the current rehabilitation program patients are offered a program at a dietitian. Usually the program is within two days and concludes a presentation and some practical exercises. Not all patients accepts this offer and therefor it would be preferable to include meal plans, diet diaries or television chefs at the platform. By this platform the patient doesn’t need to go to the dietitian, but they will be able to obtain the information through the platform. This would possible seem easier and hopefully more patients would continue a healthy lifestyle. 

By introducing telemedicine in rehabilitation both the patient and healthcare professionals would experience time savings as they doesn’t have to spent time on transportation. If telemedicine could be used to keep a healthy lifestyle for more patients, and hereby reduce readmission, a health economic reduce will also be a positive outcome. 

At Herning municipality, it is known that only 50\% of cardiac patients accepts phase 2 rehabilitation and 20\% doesn’t complete the program. It is therefore a very large group of patients who doesn’t undergo the rehabilitation program that is being offered. Hopefully by introducing telemedicine and herby a more technological and easy access rehabilitation program this large number of patients could be reduced. 

\textbf{Limitations and disadvantages} \newline
The social aspect with other patients can be hard to obtain by using telemedicine instead of cardiac classes. To most patients it gives motivation if the teamwork and dynamic between the patients is good. There has been made some qualitative research interviews which indicates that the social aspect is highly important. Furthermore, there are some ethical challenges that has to be taken into account. For example, patients must be offered the best-known rehabilitation program and as there is highly evidence that the rehabilitation program that is used today is the best this one must be the one that is being offered.     

\section{Economy and effect}

The economics aspects of this paper is conducted in collaboration with Herning municipality. The primary data for the economic analysis was gathered at Herning \textit{sportscenter} which is the location of cardiac rehabilitation in the municipality. The data was collected as an open semi-structured interview with Eva Klose Jensen, Rehabilitation and health promotion manager in Herning municipality. The data from this interview will be used for the control group resources of the standard CR treatment. The interview is to be found in App XXXX. 
%Kan vi få qualy mål fra Eva?

This case study is not a trial so there is need of data from a former conducted trial to look into the effect of implementing a ICT solution in the rehabilitation of cardiac patient. For this a study from 2015 in Belgium was chosen(Frederix et. al 2016). The set up of the trial is comparable with the danish setup. It is a randomized controlled trial to look in to the cost-effectiveness of telerehabilitaion with the ICER method. The study uses the same methodology and set up as wanted in this paper but with fokus on the Belgian system. In this paper they measure effect in Quality adjusted life years(QUALY). The result of this papers QUALY is really comparable due to Belgium and Denmark follows the same guidelines for cardiac patients. To collect the QUALY the EQ-5D questionnaire was used. in dk......
Therefor it was chosen to use the collected QUALY of the intervention group from this paper \cite{costeffect}.
The QAULY baseline in the intervention group is 0.74 and this is the number that will be used in the cost-effectiveness analysis in this paper. 

\subsection{Phase 1}

The Economic analysis in this paper does not include a detailed analysis of phase 1 which is the hospital stay of the patient. This is dismissed because the cost of the control group and intervention group is the same in this phase of the rehabilitation, furthermore the introduction of telerehabilition will not effect this fase of the rehabilitation. 


\subsection{Phase 2}

Phase 2 is the fase after the patient has been discharged. 
% Der tages ikke højde for medicamenter i analysen 
%Beskriv hvordan vi indhenter tal fra et sammenligneligt studie i belgien

\subsubsection{Control group}
% Få tabeller med skrift i midten
\begin{table}[H]
\begin{longtabu} to \linewidth{@{}l l X[j]@{}}
    \textbf{Profession} &        \textbf{Average Cost per hour} & \textbf{Total Average cost} \\[-1ex]
    \midrule
     Nurse   &    207 DKK & ? \\ \hline
    Physisican   &   184 DKK    & ? \\ \hline
    Diatrist    &     175 DKK    & ?
    \newline
   \end{longtabu}
\caption{Profession control croup cost \cite{lonnurse, lonfys, londia}}
\label{tab: PC}
\end{table}

\begin{table}[H]
\begin{longtabu} to \linewidth{@{}l l X[j]@{}}
    \textbf{Non health related} &        \textbf{Average Cost per hour} & \textbf{Total Average cost} \\[-1ex]
    \midrule
     Travel   &    ? & ? \\ \hline
     Productivity loss   &       ? & ? \\ \hline
     Presentism   &        ? & ?
    \newline
   \end{longtabu}
\caption{Non health related control croup cost}
\label{tab: NC}
\end{table}

\begin{table}[H]
\begin{longtabu} to \linewidth{@{}l l X[j]@{}}
    \textbf{Health related}  & \textbf{Total Average cost} \\[-1ex]
    \midrule
     Ambulant treatment    &    7800 DKK \\ \hline
     Other treatment   &     3022 DKK   \\ \hline
     Hospitalization   &      38594 DKK \\ \hline
     Outpatient visit   &      641 DKK \\ \hline
     GP   &      1222 DKK \\ \hline
    \newline
   \end{longtabu}
\caption{Health related control croup cost}
\label{tab: NC}
\end{table}



\subsubsection{Intervention group}

\begin{table}[H]
\begin{longtabu} to \linewidth{@{}l l X[j]@{}}
    \textbf{Profession} &        \textbf{?} & \textbf{Total Average cost/loss} \\[-1ex]
    \midrule
 	Nurse   &    207 DKK & ? \\ \hline
    Physisican   &   184 DKK    & ? \\ \hline
    Diatrist    &     175 DKK    & ?
    \newline
   \end{longtabu}
\caption{Profession intervention croup cost \cite{lonnurse, lonfys, londia}}
\label{tab: PI}
\end{table}

\begin{table}[H]
\begin{longtabu} to \linewidth{@{}l l X[j]@{}}
    \textbf{Non health related} &        \textbf{?} & \textbf{Total Average cost/loss} \\[-1ex]
    \midrule
     Travel   &    ? & ? \\ \hline
     Productivity loss   &       ? & ? \\ \hline
     Presentism   &        ? & ?
    \newline
   \end{longtabu}
\caption{Non health related intervention croup cost}
\label{tab: IC}
\end{table}

\subsection{Phase 3}



%Måske noget af det skal i bilag da det kommer til at fylde super meget


\subsection{calculations}

ICER(Cost I – Cost C)/(Effectiveness I – Effectiveness C) (DKK/QALY):











