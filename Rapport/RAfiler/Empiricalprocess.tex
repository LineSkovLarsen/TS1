\chapter{Empirical data}
\label{empirical}


The empirical data is collected from interviews both with Vibeke Lynggaard, post.doc at the Cardiovascular Clinic, Herning Hospital and PATIENT?!?!??! The data has been analyzed in section \ref{analysis}. 

%Vi skal indskrive EVA og syfoklis for økonomiske data

\section{Interview with Vibeke Lynggard, post.doc, Cardiovascular Clinic, Herning Hospital}
The interview was held as a semi structured interview and was a conversation based two-way communication. The questions was premade, but the interview was open for clarifying questions. The interview was audio-recorded and afterword the record was transcribed, see Appendix XX for the transcription.    

Vibeke Lynggaard is working as a project nurse at the clinic for cardiovascular research. The interview was also held at the cardiovascular clinic. The interview was conducted in order to gain information on telemedicine rehabilitation from a healthcare professionals point of view. Through this interview it was possible to obtain information on how telemedicine rehabilitation can be used to support the already used rehabilitation process. By the interview it was also possible to obtain an opinion on which limitations and disadvantages a telemedicine solution gives. By the information based on this interview it was possible to give an idea on how telemedicine rehabilitation can be used in the Danish healthcare system. 

\subsection{Outcome of the interview} 
Cardiac rehabilitation has been a subject to the municipalities since the 1 of January 2017. Before that it was the regions responsibility to offer rehabilitation for cardiac patients. Only one team of 12 patients are directly connected to the hospital. These patients are severely ill and therefore in need to be in contact with the hospital throughout their rehabilitation progress. Cardiac rehabilitation consists of exercise, education and social and it is important that all three collaborates. Both at the hospital and in the municipalities the patients do cardiac rehabilitation 2 times a week in 12 weeks. Furthermore, they are being educated about their illness and healthy lifestyle habits once a week. These 12 weeks of rehabilitation is described as phase 2 rehabilitation. Afterword patients are offered phase 3 rehabilitation. At the beginning of phase 2 rehabilitation the patients makes a maximal symptom-based stress test. This gives a starting point for the rehabilitation progress. Furthermore, they measure weight, height, waistline, blood pressure and heart rate. After 12 weeks they will do the test and measurements again and hereby the healthcare professionals are able to see how the patients has developed through the progress. The physiotherapist coaches the patients through a program where both cardio stress and muscle stretching are included. There are some national guidelines which indicates how much intensity and how many reputations the patients are able to do. It is necessary to keep in mind that cardiac patients struggles with different illnesses, such as reduced pump function, those who have had a new heart valve and those who has got a pacemaker. Therefore, cardiac patients got different needs and have to do different exercises.  

\textbf{How could telemedicine be included in rehabilitation?} \newline 
Phase 2 rehabilitation is mandatory to offer cardiac patients. This rehabilitation program is a class 1 recommendation from the American and European heart institutes and there is evidence that it is working. Therefore, it would be a hard process to replace phase 2 rehabilitation with telemedicine rehabilitation. All patients are being offered phase 3 rehabilitation, but most patients refuses to undergo this program. Mostly they refuse phase 3 as they want to get back to the labor market and a normal life. Unfortunately, most patients stops with common care and doesn’t continue their healthy lifestyle after the rehabilitation program. Therefore, it would be highly relevant to introduce a telemedicine solution after phase 2 rehabilitation. Furthermore, phase 3 rehabilitation is about continuing and maintaining decent exercises and dietary habits and for example continue a smoking cessation. 

More ever patients are being screened for depression at the beginning of their illness. They complete the questionnaire HADS – The Hospital Anxiety and Depression Scale. There is an algorithm behind the scale which indicates if the patient need to contact the doctor. It could be beneficial to perform a depression score once a month and hereby catch patients who is in risk of getting a depression. This questionnaire could also be a part of phase 3 and could easily be included in a telemedicine rehabilitation program. 

\textbf{Is it possible to do all measurements at home?} \newline 
Waistline can be hard to do by the patients themselves. Usually healthcare professionals do this measurement to make sure there is no bias. The objectivity disappears if the patient has to do that kind of personal measurements themselves. Furthermore, Vibeke Lynggaard mentioned that it could be relevant to connect a censor to the telemedicine platform. This could for example be used as a reminder to take medicine. Hereby the patient need to interact with the platform to let the system know that they have taken the medicine, otherwise an alarm will start. Another score that is used today is a score to indicate how nicotine dependent the patient is. The score will let the patient know whether they need to use a nicotine plaster or chewing gum. This could easily take place at the patients home as well. 

\textbf{How could telemedicine be used in rehabilitation?} \newline 
It would be beneficial to track how the patients complies with healthy lifestyle and behavior. Moreover, if the patient could interact with healthcare professional through the platform it would be very useful. By this communication platform it would be possible for the patient to ask questions about the illness and how to maintain a healthy lifestyle. For healthcare professionals it would be possible to keep an eye on the patient and keep track on their behavior. 

The rehabilitation program that is used today includes former patients to communicate experienced knowledge. This could also be included in telemedicine rehabilitation where former patients could be connected the platform and hereby it is possible for former patients and current patients to be in contact when needed.   

Furthermore, at the current rehabilitation program patients are offered a program at a dietitian. Usually the program is within two days and concludes a presentation and some practical exercises. Not all patients accepts this offer and therefor it would be preferable to include meal plans, diet diaries or television chefs at the platform. By this platform the patient doesn’t need to go to the dietitian, but they will be able to obtain the information through the platform. This would possible seem easier and hopefully more patients would continue a healthy lifestyle. 

By introducing telemedicine in rehabilitation both the patient and healthcare professionals would experience time savings as they doesn’t have to spent time on transportation. If telemedicine could be used to keep a healthy lifestyle for more patients, and hereby reduce readmission, a health economic reduce will also be a positive outcome. 

At Herning municipality, it is known that only 50\% of cardiac patients accepts phase 2 rehabilitation and 20\% doesn’t complete the program. It is therefore a very large group of patients who doesn’t undergo the rehabilitation program that is being offered. Hopefully by introducing telemedicine and herby a more technological and easy access rehabilitation program this large number of patients could be reduced. 

\textbf{Limitations and disadvantages} \newline
The social aspect with other patients can be hard to obtain by using telemedicine instead of cardiac classes. To most patients it gives motivation if the teamwork and dynamic between the patients is good. There has been made some qualitative research interviews which indicates that the social aspect is highly important. Furthermore, there are some ethical challenges that has to be taken into account. For example, patients must be offered the best-known rehabilitation program and as there is highly evidence that the rehabilitation program that is used today is the best this one must be the one that is being offered.  

\section{Interview with Hanne Voldgaard Nielsen, Professional leader, Health, Training and Assistive Technology, Herning Municipality and Eva Klose Jensen, Leader, Rehabilitation and Health, Herning Municipality}\label{sec: evahanne}

The interview was held as a semi structured interview and was conversations based between the interviewer and the two interviewees. The interview was held to give an perspective on how rehabilitation is working in Herning Municipality and to achieve information on cost-effectiveness data. The questions was premade and was based on previous and comparable cost effectiveness analysis from other similar studies \cite{costeffect}, \cite{usingeffect}. Notes were taken during the interview and can be seen in Appendix ??

Hanne Voldgaard Nielsen and Eva Klose Jensen are both working with cardiac rehabilitation in Herning Municipality. The interview was held at Holing Sport Center in Herning which is normally used for rehabilitation. The main purpose for this interview was to obtain cost-effectiveness data. By the interview it was almost possible to achieve all the cost data that was sought. The only data that wasn’t possible to get was the number of readmissions. Unfortunately, is was not possible to get any effectiveness information as the rehabilitation team does not use and calculate on that specific data. Furthermore, the interviewees gave a subjective opinion on how cardiac rehabilitation is working in Herning Municipality.              

\subsection{Outcome of the interview} 
In Herning Municipality 200 patients are being offered the rehabilitation program. \Cref{tab: reg} shows patients who did not want to participate on cardiac rehabilitation. As seen on the table the percentage of participants is very high, and only five patients turned down the offer. 

\begin{table}[H]
\begin{longtabu} to \linewidth{@{}l c c X[j]@{}}
    \textbf{Date} & \textbf{Ends before test} & \textbf{Ends after test} & \textbf{Reason} \\[-1ex]
    \midrule
     17.01.18   &    X &  & Fails to appear \\ \hline
     06.02.18   &    X &  & Dementia – tried to start the program. Does wish to continue \\ \hline
     18.04.18   &    X &  & Regretted that he said yes \\ \hline
     30.04.18   &    X &  & Called for start test - announces cancellation. Ends via non-disclosure letter \\ \hline
     23.04.18   &      & X & Depression \\ \hline
    \newline
   \end{longtabu}
\caption{Registration cardiac rehabilitation 2018}
\label{tab: reg}
\end{table}

In Herning Municipality, the rehabilitation program is offered in Sport Center Holing which is very centrally located for patients in Herning Municipality. It is known from other Municipalities that patients who lives in rural areas seem to turn down the offer as the rehabilitation program is being offered to far away from their homes. 

In Herning Municipality, the rehabilitation department agreed on marking the best possible solution where the desire was to achieve the best possible result for cardiac patients. Therefore, the solution isn’t the cheapest as the main focus isn’t on the financial part. The primary focus is on the patient and how to get them through a rehabilitation process in the best possible way with the best possible outcome. The nurses used within the rehabilitation program is hired from the cardiac department at the hospital. It is important that the nurses have the right competences to work with cardiac patients and therefore they are more experienced and on a higher salary step. The same applies for the physiotherapists. When the rehabilitation department became a subject to the municipalities the rehabilitation department had to find the right place to perform and offer the rehabilitation program. I Herning they didn’t have access to a health center and therefore they were required to invest all equipment necessary to open op a rehabilitation center. Therefore, it is important to keep in mind that the prices seen in \Cref{tab: MC} and \Cref{tab: MeC} are implementation rates and hereby a lump sum payment.   

Beside nurses and physiotherapists, a dietitian is available through the rehabilitation process. The dietitian participates four times a year for three hours. Furthermore, cardiac patients are being offered smoking cessation during and after the rehabilitation program. Most patients do not have the time and energy to undergo a smoking cessation while they participate in the rehabilitation program. Therefore, it is possible to do a smoking cessation after the 12 weeks of rehabilitation and it will still be paid by the municipality.   

An important part in the rehabilitation program is that a previous patient participates in the program. By this concept patients are able to share experiences and to get knowledge on how to deal with a chronical disease. This is a very important factor to many of the patients as it is very trustworthy to hear personal disease and health experiences. 

By the interview it was mentioned that patients are being pressed hard physically to obtain the best result. A few times it has been experienced that some patients feel ill and are in need for personal help and assistance. As this patient group suffer from heart diseases it’s important to keep in mind that they are in risk of getting a heart attack while they are doing their exercises. This is an important information to be aware off if telemedicine is introduced in rehabilitation. If patients are doing their exercise via telemedicine at home instead of participating at a team at the rehabilitation center, they are not able to receive the same urgent help if a heart attack or another emergency health situation appears.         

\section{Economy and effect}

The economics aspects of this paper is conducted in collaboration with Herning municipality. The primary data for the economic analysis was gathered at Herning \textit{sportscenter} which is the location of cardiac rehabilitation in the municipality. The data was collected as an open semi-structured interview with Eva Klose Jensen, Rehabilitation and health promotion manager in Herning municipality and Hanne Volsgaard Nielsen, Professional leader at Herning municipality. The data from this interview will be used for the control group resources of the standard CR treatment. The interview is to be found in App XXXX. 


This case study is not a trial so there is need of data from a former conducted trial to look into the effect of implementing a ICT solution in the rehabilitation of cardiac patient. For this a study from 2015 in Belgium was chosen(Frederix et. al 2016). The set up of the trial is comparable with the danish setup. It is a randomized controlled trial to look in to the cost-effectiveness of telerehabilitation with the ICER method. The study uses the same methodology and set up as wanted in this paper but with focus on the Belgian system. In this paper they measure effect in Quality adjusted life years(QALY). The result of this papers QALY is really comparable due to Belgium and Denmark follows the same guidelines for cardiac patients that is provided from the European Union. To collect the QALY the EQ-5D questionnaire was used before, after and under the trial. Therefor it was chosen to use the collected QALY of the intervention group from this paper. Herning municipality do not collect effectiveness data on their patient and due to the lack of data for the control group the QALY of the control group is used from the Belgian project \cite{costeffect}.


%beskriv control gruppe og intervention gruppe
%Beskriv at priserne er implementerings prisen


\subsection{Phase 1}

The Economic analysis in this paper does not include a detailed analysis of phase 1 which is the hospital stay of the patient. This is dismissed because the cost of the control group and intervention group is the same in this phase of the rehabilitation, furthermore the introduction of telerehabilitation will not effect this fase of the rehabilitation. 


\subsection{Phase 2}

Phase 2 is the fase after the patient has been discharged. 
% Der tages ikke højde for medicamenter i analysen 
%Beskriv hvordan vi indhenter tal fra et sammenligneligt studie i belgien





\subsubsection{Control group}

\textbf{Economics}
All cost for the control group was collected during the semi-structured interview with Eva and Hanne described in \cref{sec: evahanne}. 

The cost of profession is the first cost that is taken in to count.
The salary and working hours of the professions needed in the control group is based on the cost in Herning municipality this means it will differ from municipality too municipality. The collected data will be described further in this section where cost per patient will be the outcome of the calculations. 


The salary of the nurses needed in the standard CR is 308DKK per hour and they have two nurses who works 21 hours, 48 weeks a year. The cost per patient will there for be.

Total hours of working nurses a year:

$$42hours\cdot48weeks=2016hours$$

Total cost per year:
$$2016hours\cdot308DKK=620.928DKK$$

Cost per patient:
$$\frac{620.928DKK}{200patients}=3104,64DKK$$

The cost per year for the physiotherapist is 600.000DKK hence the calculations for the physiotherapist is backwards. The physiotherapist works 54 hours a week in 48 weeks a year. The cost of the physiotherapist is therefor.

Total hours of working physiotherapist a year:
$$54hours\cdot48weeks=2592hours$$

Physiotherapist hourly cost:

$$\frac{600.000DKK}{2592hours}=231,50DKK$$

Cost per patient:
$$\frac{600.000DKK}{200patients}=3000DKK$$

The salary of the dietician needed in the standard CR is 225DKK per hour and the nurse is on education schedule four times a year for three hours. Furthermore the dietician get paid for two hours of preparation per class.  Therefor the cost of the dietician is.

Total hours of working dietician a year:

$$5hours\cdot4=20hours$$

Total cost per year:
$$20hours\cdot225DKK=4500DKK$$

Cost per patient:
$$\frac{4500DKK}{200patients}=22,5DKK$$

\begin{table}[H]
\begin{longtabu} to \linewidth{@{}l l l X[j]@{}}
    \textbf{Profession} & \textbf{Average hourly Cost} & \textbf{Total cost a year} & \textbf{Cost per patient} \\[-1ex]
    \midrule
     Nurse   &    308 DKK & 620.928DKK & 3104,64DKK \\ \hline
    Physiotherapist   &   231,50DKK  & 600.000DKK & 3000DKK \\ \hline
    Dietician   &  225DKK &    4500 DKK    & 22.5DKK \\ 
    \hline \hline \hline
    \textbf{Total} & & 1.225.428DKK & 6127,14DKK
    \newline
   \end{longtabu}
\caption{Profession control group cost}
\label{tab: PC}
\end{table}

The next cost that is considered is the implementation cost of materials as training equipment and office environment at the training facility. It is important to notice that these cost is a one time cost and in the future it will be a significant minor cost. The data is collected from a spreadsheet from Herning municipality. The numbers a calculated volume of this spreadsheet to find the details on what the different headlines contains please read appendix XXXX. 

\begin{table}[H]
\begin{longtabu} to \linewidth{@{}l l l X[j]@{}}
    \textbf{Materials implementation} & \textbf{Unit} &\textbf{Total Cost} & \textbf{Cost per patient} \\[-1ex]
    \midrule
    Training equipment   &  97 &  29.711DKK & 149DKK \\ \hline
    Training bikes   & 12 & 118.200DKK & 591DKK  \\ \hline
    Test bike    &  1 & 76.750DKK &   382.75DKK \\ \hline 
    Office environment    &  39 & 34.545DKK  &   173DKK  \\ \hline 
    Other material   &  8 & 18.161DKK  &   91DKK\\
    \hline \hline \hline
    \textbf{Total} & 157 & 277.367DKK & 1386,75DKK
    \newline
   \end{longtabu}
\caption{Materials control group cost}
\label{tab: MC}
\end{table}

To provide the right care for the patients and follow up on their condition some medical equipment is needed. These are described in \cref{tab: MeC}. The data on this is from the spreadsheet as before mentioned and is to be seen in appendix XXX.

\begin{table}[H]
\begin{longtabu} to \linewidth{@{}l l l X[j]@{}}
    \textbf{Medical equipment} & \textbf{Unit} &\textbf{Total Cost} & \textbf{Cost per patient} \\[-1ex]
    \midrule
    Model of heart   &  1 &  350DKK & 1,75DKK \\ \hline
    Sphygmomanometer  & 1 & 1599DKK & 8DKK  \\ \hline
    Pulse Oximeter    &  1 & 599DKK &   3DKK \\ \hline 
    Cuff    &  2 & 498DKK  &   2,5DKK  \\ \hline 
    Ventilation mask   &  1 & 1637DKK  &   8DKK \\
    \hline \hline \hline
    \textbf{Total} & 6 & 4683DKK & 23,25DKK
    \newline
    \newline
   \end{longtabu}
\caption{Medical equipment control group cost}
\label{tab: MeC}
\end{table}

The rest of Herning Kommunes cost is collected in this last table. It is other values and it contains cost as rent for their training facility and education of employees. The Cost of renting the facility is not given in this due to a non disclosure agreement with Herning municipality. The group know the value of the cost and is able to tell that is a big cost for the municipality, but it is not possible to publish the number in this paper. The number of rehospitalized patients in Region Midtjylland was 4658 in 2015 out of 20289 admissions, hence the percentage of rehospitalization per patient is 23\% in the Region. Though the total number os readmissions is 6339, which means a total percentage of 31\% readmissions \cite{hjertetal}. It not possible to collect the exact data from Herning municipality. The total cost is calculated as that is expected that 23\% of all patients in CR treatment in Herning municipality i rehospitalized. The average cost of rehospitalisation was 100.875DKK in 2004 \cite{rasmussen2011hjerterehabilitering}. This is the newest number to collect, but the cost is higher today, but unknown. The calculation of cost in Herning municipality is as followed.

Estimated amount of readmittions to hospital in Herning:

$$31\%200= 62 readmission$$

Total cost:
$$62\cdot100.875DKK=6.254.250DKK$$

Total cost spread out as a cost on all patients:

$$\frac{6.254.250DKK}{200}=31.271,25DKK$$


\begin{table}[H]
\begin{longtabu} to \linewidth{@{}l l l X[j]@{}}
    \textbf{Other Cost} &\textbf{Total Cost a year} & \textbf{Cost per patient} \\[-1ex]
    \midrule
    Location   &  (NDA) & (NDA) \\ \hline
    Employee education   & 10.000DKK & 50DKK  \\ \hline
    HjerteKomMidt (database)  & ?DKK &   ?DKK \\ \hline
    Brochure & 0DKK & 0DKK \\ \hline
    Rehospitalization & 6.254.250DKK & 31.271,25DKK \\
    \hline \hline \hline
    \textbf{Total} & ?  & ?DKK
    \newline
    \newline
   \end{longtabu}
\caption{Other cost control group}
\label{tab: OC}
\end{table}



From these collected cost from Herning it is possible to calculate the total cost per patient in the based on the first year after implementation, this indicates that the fixed implementation cost is included is this analysis.

$$Profession + Material implementation + Medical equipment + other cost = Cost per patient$$
% Her skal lige være summering af priser.

\textbf{Effectiveness}

 The QALY baseline in the control group is 0.77 and this is the number that will be used in the cost-effectiveness analysis in this paper\cite{costeffect}. 

\subsubsection{Intervention group}

\begin{table}[H]
\begin{longtabu} to \linewidth{@{}l l l X[j]@{}}
    \textbf{Profession} & \textbf{Average hourly Cost} & \textbf{Total cost a year} & \textbf{Cost per patient} \\[-1ex]
    \midrule
     Nurse   &    308 DKK & 620.928DKK & 3104,64DKK \\ \hline
    Physiotherapist   &   231,50DKK  & 600.000DKK & 3000DKK \\ \hline
    Dietician   &  225DKK &    4500 DKK    & 22.5DKK \\ 
    \hline \hline \hline
    \textbf{Total} & & 1.225.428DKK & 6127,14DKK
    \newline
   \end{longtabu}
\caption{Profession Intervention group cost}
\label{tab: PI}
\end{table}

\begin{table}[H]
\begin{longtabu} to \linewidth{@{}l l l X[j]@{}}
    \textbf{Materials implementation} & \textbf{Unit} &\textbf{Total Cost} & \textbf{Cost per patient} \\[-1ex]
    \midrule
    Training equipment   &  97 &  29.711DKK & 149DKK \\ \hline
    Training bikes   & 12 & 118.200DKK & 591DKK  \\ \hline
    Test bike    &  1 & 76.750DKK &   382.75DKK \\ \hline 
    Office environment    &  39 & 34.545DKK  &   173DKK  \\ \hline 
    Other material   &  8 & 18.161DKK  &   91DKK\\
    \hline \hline \hline
    \textbf{Total} & 157 & 277.367DKK & 1386,75DKK
    \newline
   \end{longtabu}
\caption{Materials intervention group cost}
\label{tab: MI}
\end{table}

\begin{table}[H]
\begin{longtabu} to \linewidth{@{}l l l X[j]@{}}
    \textbf{Medical equipment} & \textbf{Unit} &\textbf{Total Cost} & \textbf{Cost per patient} \\[-1ex]
    \midrule
    Model of heart   &  1 &  350DKK & 1,75DKK \\ \hline
    Sphygmomanometer  & 1 & 1599DKK & 8DKK  \\ \hline
    Pulse Oximeter    &  1 & 599DKK &   3DKK \\ \hline 
    Cuff    &  2 & 498DKK  &   2,5DKK  \\ \hline 
    Ventilation mask   &  1 & 1637DKK  &   8DKK \\
    \hline \hline \hline
    \textbf{Total} & 6 & 4683DKK & 23,25DKK
    \newline
    \newline
   \end{longtabu}
\caption{Medical equipment Intervention group cost}
\label{tab: MeI}
\end{table}

\begin{table}[H]
\begin{longtabu} to \linewidth{@{}l l l X[j]@{}}
    \textbf{Other Cost} &\textbf{Total Cost} & \textbf{Cost per patient} \\[-1ex]
    \midrule
    Location   &  (NDA) & (NDA) \\ \hline
    Employee education   & 10.000DKK & 50DKK  \\ \hline
    HjerteKomMidt (database)  & ?DKK &   ?DKK \\ \hline
    Brochure & 0DKK & 0DKK \\ \hline
    Rehospitalization & ? & ? \\
    \hline \hline \hline
    \textbf{Total} & ?  & ?DKK
    \newline
    \newline
   \end{longtabu}
\caption{Other cost Intervention group}
\label{tab: OI}
\end{table}

\textbf{Effectiveness}

The QALY baseline in the intervention group is 0.74 and this is the number that will be used in the cost-effectiveness analysis in this paper\cite{costeffect}. 

\subsection{Phase 3}


\subsection{calculations}

ICER(Cost I – Cost C)/(Effectiveness I – Effectiveness C) (DKK/QALY):
%Måske noget af det skal i bilag da det kommer til at fylde super meget















