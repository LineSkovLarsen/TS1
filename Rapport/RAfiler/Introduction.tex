\chapter{Introduction}

\section{Background}
The background section will give a short introduction to the three main areas in this project; The technology that is used in vCare, the Danish Healthcare System and finally an introduction to the main target group who will be affected when introducing telemedicine in rehabilitation. 

\subsection{Digitalization within the Danish Healthcare System}

Denmark is known for extensive digitization and electronic communication within the Healthcare System and the use of health data. Denmark made standards for electronic communication years ago and the result of this is an almost digitalized communication within the healthcare sector. Health records, laboratory test results and hospital referrals are all nearly collected as electronic data. 

Multiple ICT and digital workflow are completely integrated, this marks Denmark as a frontrunner in deployment of e-health. Telemedicine is a big part of the digitalization plan in Denmark where five initiatives is to be the foundation of future telemedicine infrastructure in Denmark. "The goal is to have a digital infrastructure and IT architecture in place within the foreseeable future, so that relevant information can be exchanged across the healthcare system and other sectors" \cite{Healthcareindk2}.

In 2011 Denmark started a project for telemedicine throughout the country. The five regions in Denmark made a strategy on how to develop telemedicine in a wider scale and combine it with effective shared knowledge. For this to happen a board has been chosen and is known as the National Board of E-Health \cite{DKhealthreview}. 

\subsection{Technology}

The technology section will be based on the new ICT based virtual coaching solution vCare. 

The basic concept of vCare is carried out by a central eHealth platform that serves central infrastructure services. The platform obtains the information delivered by sensors or gained by the direct interaction between the patient and the virtual coach. The devices added to this platform consist of camera, microphones and a Kinect, which makes the platform able to track movements. The information from the devices are conducted by a real-time processor. Beside the platform the infrastructure delivers supporting services to improve the quality of life of patients. The service provides physical and cognitive exercises as well as education material within nutrition and life behaviour. This service will be extended with a care pathway and a knowledge layer that enables personalized exercises and material for the given patient. Based on algorithms the virtual coach is flexible regarding the patients’ needs and hereby able to make specific rehabilitation programs. The platform can be implemented on different devices, e.g., tablets, smartphones, TV screens etc. \cite{Technical}. 



\subsection{Target Group and Market Segment}

By introducing telemedicine, the rehabilitation process is brought directly to patients' homes and mostly targets people with chronic conditions, which includes cardiac patients. Telemedicine rehabilitation is used to prevent hospitalization, to improve patients' feeling of safety, to empower patients to manage their own chronic condition and hereby improve patients' quality of life \cite{Emergence}. 

The need for cardiac rehabilitation is evaluated for all patients with heart disease. This includes both patients who have had a balloon dilation or by-pass surgery and patients with stable ischemic heart disease.
Patients with heart failure, pacemaker or patients who have had heart-valve surgery or cardiac transplantation are also being evaluated for the purpose of cardiac rehabilitation \cite{Rehabilitering}. By this statement it is seen, that this invention will involve a large target group.

To teach cardiac patients about their illness and how they are able to influence the course of the disease, results in a reduced risk of dying. Furthermore, research shows that rehabilitation programs with physical exercises results in reduced cardiac mortality \cite{Hjerteforening}.    
   
\newpage
\section{Problem Statement}
More than half of the danish citizens over the age of 55 suffer from a cardiovascular disease. Furthermore, cardiovascular diseases are one of the most common causes to death in Denmark. The total cost of treating cardiovascular patients at the Danish Healthcare System was 5.5 billion DKK in 2015. Every year approximately 55.700 Danes is diagnosed with cardiovascular disease.   

Nearly 107.100 Danes are hospitalized every year for cardiovascular disease and almost 73.100 Danes are yearly at one or more consultations at the hospital. Approximately 23 percent of the cardiovascular patients are readmitted into the hospital within 30 days after being discharged. It has been proven, that cardiac rehabilitation results in a reduction in mortality caused by cardiovascular diseases as well as the need for readmissions \cite{Hjerteforening}.

All this indicates that cardiovascular patients constitute a large part of the Danish states economy. This leads to our problem statement which is:

\begin{itemize}
	\item What impact would an ICT solution for rehabilitation have on both cardiovascular patients and the Danish Healthcare System?
	\item How can ICT help to prevent readmission of cardiac patients?
	\item Which barriers/challenges can such system meet in implementation?
\end{itemize}

\subsection{Delimitation}
This project is limited only to be focusing on healthcare in Denmark, mainly focusing on Herning Municipality, and how the technology within rehabilitation will have an essential impact on the Danish Healthcare System. However, the project will be compared to related ICT solutions in EU as scientific articles based on The Danish Healthcare System is limited in this research area. 

Relevant data on how the Danish Healthcare System is establish will mainly be based on literature found in books and on websides where guidelines, statistics and the historical development is being published. 


