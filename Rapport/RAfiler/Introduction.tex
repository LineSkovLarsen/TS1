\chapter{Introduction}

\section{Background}
\subsection{Technology}
\subsection{The Danish Healthcare System}
\subsection{Target Group and Market Segment}

The need for cardiac rehabilitation should be evaluated for all patients with heart disease. This includes patients who have had a balloon dilation or by-pass surgery and patients with stable ischemic heart disease.
Patients with heart failure, pacemaker or who have had heart-valve surgery or cardiac transplantation should also be evaluated for the purpose of cardiac rehabilitation \cite{Rehabilitering}. By this statement it is seen, that this invention will involve a large target group. 

To teach cardiac patients about their illness and how they are able to influence the course of the disease, means that the risk of dying is reduced. Furthermore research shows, that rehabilitation programs with physical exercise reduce cardiac mortality \cite{Hjerteforening}.    


\section{Problemstatement}
More than half of the danish citizens over the age of 55 suffer from a cardiovascular disease. Furthermore, cardiovascular diseases are one of the most common causes to death in Denmark. The total cost of treating cardiovascular patients at the Danish Healthcare System was 5.5 billion DKK in 2015. Every year approximately 55.700 Danes is diagnosed with cardiovascular disease.   

Nearly 107.100 Danes are hospitalized every year for cardiovascular disease and almost 73.100 Danes are yearly at one or more consultations at the hospital. Approximately 23 percent of the cardiovascular patients are readmitted into the hospital within 30 days after being discharged. It has been proven, that cardiac rehabilitation results in a reduction in deaths caused by cardiovascular diseases and the need for readmissions \cite{Hjerteforening}.

All this indicates that cardiovascular patients constitute a large part of the Danish states economy. This leads to our problem statement which is:

\begin{itemize}
	\item What impact would an ICT solution for rehabilitation have on both cardiovascular patients and the Danish Healthcare System?
	\item How can ICT be used to shorten hospital stay for cardiovascular patients?
	\item Which barriers/challenges can such system meet in implementation?
\end{itemize}

\subsection{Delimitation}
This project is limited only to be focusing on healthcare in Denmark and how the technology within rehabilitation will have an essential impact on the Danish Healthcare System. However, the project will be compared to related ICT solutions in EU as scientific articles based on The Danish Healthcare System is limited in this research area. 

Relevant data on how the Danish Healthcare System is establish will mainly be based on literature found in books and on websides were guidelines, statistics and the historical development is being published. 


