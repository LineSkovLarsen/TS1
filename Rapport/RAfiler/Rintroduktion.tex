\chapter{Introduktion}

En forskningsgruppe fra CAVE lab på Aarhus Universitet, Institut for Ingeniørvidenskab, forsker i samarbejde med Kardiologisk Afdeling på Aarhus Universitetshospital, Skejby, i en ny metode indenfor interventionel kardiologi, Balloon Fracturing before Valve-in-Valve (BF-ViV). 

Valve-in-Vale (ViV) er en allerede anvendt interventionel behandlingsmetode, som bliver tilbudt og ofte foretrukket til reoperation af degenererede biologiske hjerteklapper hos højrisiko patienter med aortastenose og insufficiens. Metoden har vist sig at have sine begrænsninger hos patienter med degenererede biologiske hjerteklapper $\leq21$ mm. Den effektive klapsåbning ved disse patienter vil efter ViV blive yderligere reduceret og dermed forårsage en for stor belastning for hjertet.

Med metoden BF-ViV ønskes det at udvide behandlingsmulighederne til disse patienter. Princippet ved metoden er at frakturere den allerede implanterede degenererede biologiske hjerteklap. Dette gøres med henblik på at udvide den effektive klapåbning, så en ny og større biologisk hjerteklap kan opsættes. Indtil nu har forskningen vist, at det nødvendige frakturtryk varierer afhængig af type af den biologiske hjerteklap, ballonstørrelse samt tilstedeværelsen af kalk på aortaroden.  

I dette bachelorprojekt vil fokus være at undersøge, hvorvidt forskellige kalcificeringsgrader af aortarodens annulus, har betydning for det nødvendige fraktureringstryk. Ydermere vil fokus være at udvikle en prototype af et system, der kan detektere frakturering af den degenererede biologiske hjerteklap. Detekteringssystemet skal støtte og bekræfte operatøren i, at fraktureringen har fundet sted og dermed gøre BF-ViV proceduren mere sikker og præcis.     



      

