\chapter{Methodology}

%How am I going to do this?
This study is a combined multi-method qualitative case study divided in three fases.
The study is combined between exploratory, descriptive and evaluative study. Each study type is represented in a research question. Furthermore the study is a multi-method qualitative study due to two types qualitative analyzing techniques. 
\section{Research design}

% Indsæt billede når det er færdigt

 Fase 1 in the study is the initial process, fase 2 is knowledge/data and fase 3 is analysis/outcome. 
In fase 1 the topic was selected and translated into questions and hereby the problem statement. In this process the delimitation of the project was laid out. Furthermore the methodology was chosen during this fase in the project. The methodology contains considerations in literature search, interview method and data analyzing methods. 
Fase 2 is the process to gain knowledge and collect the necessary data to analyze the problem. The knowledge is gathered through a literature search in the area. The method is described in \cref{literature}. The data is collected throughout interviews with both cardiac patients and a research nurse. To learn more about the interview and the empirical process read \cref{qualitative} and \cref{empirical}.
In fase 3 the collected data and knowledge is analyzed and evaluated. Furthermore the newly gathered information are discussed within the literature search. The closing statement in this study will be an overall conclusion of the studies findings. The last process in the project is to reflect and look into further investigation of the problem. 
%Mono method qualitative study eller multimethod... alt efter om vi får data fra patienter


\section{Literature search method}
\label{literature}

The literature was conducted with a thorough literature research. To find the right type of literature PICO(Population/Problem, Intervention, Comparison/Control, Outcome) was used as a framework, see \cref{PICO}. 
Following databases have been used for this project: PubMed, AUlibrary, Embase, google scholar, % Er der flere?
The literature search started in february 2018 where the primary part of the literature was collected within two month, although literature has been collected through the hole period of writing the project. 
Search keywords was conducted in the problem statement and have been used for search words in the databases. Following words was chosen as key words: ICT, Healthcare, cardio-vascular, rehabilitation, cost effectiveness.
The papers were chosen from title and abstract. Furthermore other literature was conducted through a chained search in relevant literature. Multiple papers was deselected due to irrelevance or mismatch of the subject. The national guidelines and national history was conducted on state webpages. The keywords were combined with an "AND" - and in related areas as "OR". The PICO blocks was as well combined with the "And" and "OR". % skal vi have noget om hvilket år artiklen er skrevet som afgrænsning

\begin{table}[H]
\begin{longtabu} to \linewidth{@{}l l X[j]@{}}
    \textbf{PICO} &        \textbf{Search headings} \\[-1ex]
    \midrule
     Population/Problem:   &    Patients with cardiac illness \\ \hline
    Intervention:   &        Patient in telerehabiliation \\ \hline
    Comparison/Control:    &        Standard cardiac rehabilitation \\ \hline
    Outcome:    &        Adherence to CR, readmission, mortality 
    \newline
   \end{longtabu}
\caption{Search headings in PICO principles}
\label{PICO}
\end{table}



\section{Research interviews}
\label{qualitative}

Research interview is defined as a purposeful conversation between two or more people whereas an interviewer ask concise and unambiguous questions and the interviewee will respond. To collect data for this study interviews with experts in rehabilitation of  cardiac patients and cardiac patients themselves have been made. 
There is many types of interview but for this study Semi-structured interview has its purpose. The semi-structured interviews is exploratory, explanatory and evaluative. Furthermore this kind og interview is referred to as qualitative research interview.
This type of interview makes it possible to make a frame of the interview but it also allow the interviewee to expand the knowledge area end hereby expand the frame of the interview. Besides the expansion of the frame this setup allows the interviewee to explain the opinions and reason for attitude. Semi-structured interview provides the opportunity to probe answers whereas the interviewee can explain their responses. The interview may also lead the discussion in to unforeseen areas, which can collaborate to new knowledge. The interview types gives a detailed set of data but it can be viewed as biased due to the interviewers impact on the interviewee. 
In this project the questions are formed as open ended and only as a frame hence the semi-structured interview.

To prepare for the semi-structured interview we used the "five p's": Prior planning prevents poor performance. To withhold these p's following was taken in to count. Level of knowledge, developing the interview themes, inform interviewee before interview and finding an appropriate location. 
The group had gained a lot of prior knowledge of the rehabilitation of cardiac patient in Denmark before the interview which supports the capability to accurate respons in the interview and the interviewers credibility. The knowledge was secured during the literature research fase. 

For the interview with the expert in rehabilitation of cardiac patients some interview questions were designed to make sure that the every area wanted was conducted although the nurse was able to walk outside the framework. The question ideas came from read literature and the problem statement. The prepared questions is present in app XXX. If the interview do not have some focus the interview might lack a sense of purpose. The frame made for this intervies was made like a guide in a perhaps logical order. The location of the interview should be convenient for the partipant otherwise they might feel uncomfortable which could impact the data collection. For this interview the participant chose the location to oblige convenience for the participant. 
 %Skal der være noget om hvilken type spørgsmål der blev stillet?
 %skal vi mere i dybden ?

interview patienter
% vi skal være meget opmærksomme på bias i vores diskussion af data collection brug KAspers bog til sætte de korrekte ord på. 

\section{Analyzing qualitative data}


Qualitative research depend on social interaction, hence the analyzing qualitative data in an interactive and iterative process. Qualitative data are likely to be more varied, elastic and complex than quantitate. A analyzing method is therefor a great tool to evaluate and use the data to answer the research questions. 

To analyze the interview with the research nurse an analyzing tool is necessary. The Narrative analysis method has been chosen to this. The narrative analysis consist of a collection of different approaches to analyse qualitative data. 
 The study only have the narrative of one individual but the nurse will give another perspective into the healthcare system than state literature can. She gives the opportunity to look into a small peace of the danish healthcare system, more likely region Midtjylland and specific Herning hospital. Due to the interview only being collected on one individual coding has been deselected in this analysis.  

%Laver vi deduktion eller induktion
%analyse metode patienter

\section{Economy}

%Hvad koster et fuldt forløb for en normal cardiac patient fra fase 1 til 3
%Hvad vil en patient koste fra indlæggelse til udfasing med tele health

Various approaches are possible when valuing and comparing a technology with the existing method. For this study a few approaches has been chosen to look into the cost and effect of telerehabilitation in Denmark.

\subsection{Cost-effectiveness analysis}



When implementing new technology and rehabilitation processes it is important to look at the cost. Every region and municipality is on a budget provided by the state. Therefor resource allocation is a big part of the Danish system. But in health care cost is not the only value taken in to account, the patients health is an important part of the puzzle. When allocating money for one intervention another intervention may be dismissed. This is why a decision tool is needed to evaluate interventions and pick the interventions that provides the most benefit with the available resources.
Cost-effective analysis(CEA) is a an analysis of cost and effectiveness of perhaps a new service or technology. The benefit if using a CEA is that it does not only look in to the cost but also takes the patient in to count. It is often often used to evaluating effectiveness in healthcare. In this study the CEA is used to compare the traditional CR with a the new setup with telerehabilation and is therefor the Incremental Cost-Effectiveness Ratio(ICER) type of CEA. In the analysis the CE ratio is calculated. The CE ratio is the cost associated divided with health outcome.
The data for the analysis is collected in cooporation with \textit{Herning sygehus} and Herning municipality. 

Following cost are taken in to count:

\begin{longtabu} to \linewidth{@{}l X[j] @{}}
	\textbf{Health related} & \textbf{Non-health related}\\[-1ex]
	\midrule
	Staff & Travel \\[-1ex]
	Outpatient visits & Productivity loss \\[-1ex]
	Paramedical visits & Presentism loss\\[-1ex]
	GP visits &  \\[-1ex]
	Monitoring devices &  \\[-1ex]
	Medication &  \\[-1ex]
	Hospitalization &  \\[-1ex]
	\hline
	\caption{Cost variables CEA}
\end{longtabu}


effect meassure:
PAL or DALY or QALY

% Vi skal skrive noget om hvordan de forskellige parametre indsamles, jeg tror det kræver input fra Albena 

%Fra netherlands paper:
%Prices  published  in the  Dutch  Manual  for  Costing  in  economic  evaluations and  market  prices  are  used  [42].  Other  costs  are  mea-sured  using  the  modified  iMCQ  (Medical  ConsumptionQuestionnaire)  and  iPCQ  (Productivity  Cost  Questionnaire)  questionnaires  [40,  41],  which  patients  fill  out  atthree,  six,  nine  and  twelve  months.  To  assess  the  effects of  the  delivery  of  informal  care  and  to  include  these  in the  economic  evaluation,  informal  carers  of  patients  fill out  the  minimum  variant  of  the  iVICQ  (Valuation  of  In-formal  Care  Questionnaire)  at  three,  six,  nine  and  twelvemonths  [43].  Either  the  friction  cost  method  or  the  hu-man  capital  approach  is  applied  to  determine  the  costs of  absenteeism  or  inefficiency  from  paid  work  [44].

%Man kan også se på hvor mange der ville gennemføre forløbet hvis det foregik i hjemmet
% man kan sætte det i en graf  hvor CE ratio kan visualiseres se youtube video

%Mulige måle metoder
%cost-benefit...
%cost-utility 
%measures in Qualy 
%cost-impact






