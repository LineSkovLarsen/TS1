\chapter{Method}

%How am I going to do this?

\section{Research design}

% Indsæt billede når det er færdigt

This study is divided in three fases the initial fase 1, knowledge/data collection fase 2 and analysis/outcome fase 3. 
In fase 1 the topic was selected and translated into questions and hereby the problem statement. In this process the delimitation of the project was laid out. Furthermore the methodology was chosen during this fase in the project. The methodology contains considerations in literature search, interview method and data analyzing methods. 
Fase 2 is the process to gain knowledge and collect the necessary data to analyze the problem. The knowledge is gathered through a literature search in the area. The method is described in \cref{literature}. The data is collected throughout interviews with both cardiac patients and a research nurse. To learn more about the interview and the empirical process read \cref{qualitative} and \cref{empirical}.
In fase 3 the collected data and knowledge is analyzed and evaluated. Furthermore the newly gathered information are discussed within the literature search. The closing statement in this study will be an overall conclusion of the studies findings. The last process in the project is to reflect and look into further investigation of the problem. 

\section{Literature search method}
\label{literature}

The literature was conducted with a thorough literature research. To find the right type of literature PICO(Population/Problem, Intervention, Comparison/Control, Outcome) was used as a framework, see \cref{PICO}. %Indsæt reference til tabel
Following databases have been used for this project: PubMed, AUlibrary, Embase, google scholar, % Er der flere?
The literature search started in february 2018 where the primary part of the literature was collected within two month, all though literature has been collected through the hole period of writing the project. 
Search keywords was conducted in the problem statement and have been used for search words in the databases. The papers were chosen from title and abstract. Furthermore other literature was conducted through a chained search in relevant literature. Multiple papers was deselected due to irrelevance or mismatch of the subject. The national guidelines and national history was conducted on state webpages. The keywords were combined with an "AND" - and in related areas as "OR". The PICO blocks was as well combined with the "And" and "OR". % skal vi have noget om hvilket år artiklen er skrevet som afgrænsning

\begin{table}[H]
\begin{longtabu} to \linewidth{@{}l l X[j]@{}}
    \textbf{PICO} &        \textbf{Search headings} \\[-1ex]
    \midrule
     Population/Problem:   &    Cardiac patients \\ \hline
    Intervention:   &        Telerehabilitation \\ \hline
    Comparison/Control:    &        ?? \\ \hline
    Outcome:    &        CR, readmission, mortality 
    \newline
   \end{longtabu}
\caption{Search headings in PICO principles}
\label{PICO}
\end{table}



\section{Research interviews}
\label{qualitative}

Research interview is defined as a purposeful conversation between two or more people whereas an interviewer ask concise and unambiguous questions and the interviewee will respond. To collect data for this study interviews with experts in rehabilitation of  cardiac patients and cardiac patients themselves have been made. 
There is many types of interview but for this study Semi-structured interview has its purpose. The semi-structured interviews is exploratory, explanatory and evaluative. Furthermore this kind og interview is referred to as qualitative research interview.
This type of interview makes it possible to make a frame of the interview but it also allow the interviewee to expand the knowledge area end hereby expand the frame of the interview. Besides the expansion of the frame this setup allows the interviewee to explain the opinions and reason for attitude. Semi-structured interview provides the opportunity to probe answers whereas the interviewee can explain their responses. The interview may also lead the discussion in to unforeseen areas, which can collaborate to new knowledge. The interview types gives a detailed set of data but it can be viewed as biased due to the interviewers impact on the interviewee. 
In this project the questions are formed as open ended and only as a frame hence the semi-structured interview.

To prepare for the semi-structured interview we used the "five p's": Prior planning prevents poor performance. To withhold these p's following was taken in to count. Level of knowledge, developing the interview themes, inform interviewee before interview and finding an appropriate location. 
The group had gained a lot of prior knowledge of the rehabilitation of cardiac patient in Denmark before the interview which supports the capability to accurate respons in the interview and the interviewers credibility. The knowledge was secured during the literature research fase. 

For the interview with the expert in rehabilitation of cardiac patients some interview questions were designed to make sure that the every area wanted was conducted although the nurse was able to walk outside the framework. The question ideas came from read literature and the problem statement. The prepared questions is present in app XXX. If the interview do not have some focus the interview might lack a sense of purpose. The frame made for this intervies was made like a guide in a perhaps logical order. The location of the interview should be convenient for the partipant otherwise they might feel uncomfortable which could impact the data collection. For this interview the participant chose the location to oblige convenience for the participant. 
 %Skal der være noget om hvilken type spørgsmål der blev stillet?
 %skal vi mere i dybden ?

interview patienter
% vi skal være meget opmærksomme på bias i vores diskussion af data collection brug KAspers bog til sætte de korrekte ord på. 

\section{Analyzing qualitative data method}

\section{economy}


cost-benefit...









