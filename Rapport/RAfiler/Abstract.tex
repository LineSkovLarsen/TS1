\chapter{Abstract}

\textbf{Introduction} \newline
Cardiac rehabilitation is a common term for post treatment of patients suffer from chronic cardiac disease. It is currently recommended as a part of the overall treatment for cardiac patients. Benefits of cardiac rehabilitation are well established and is has been proven that cardiac rehabilitation has positive effects on morbidity and mortality in cardiac patients. Unfortunately, long-term sustainability of exercise effects and the key to continue a healthy lifestyle are unsolved issues and tele rehabilitation could have the potential to overcome this. Experience has shown that telemedicine can be used to effective the public healthcare service and hereby economic and financial aspects could benefit from telemedicine solutions.  

\textbf{Method} \newline
The purpose of this project was to look further into the Danish Healthcare System and how the system could be affected by a telemedicine solution within cardiac rehabilitation. More specific Herning Municipality has been used as the main area of interest. To gain knowledge a literature research has been performed. Literature upon this field is mainly based on telemedicine solutions within other diseases areas or studies where cardiac telemedicine rehabilitation has been evaluated as a trial in other countries. To achieve information on cardiac rehabilitation in Denmark, some interviews has been conducted whereas healthcare professionals from Herning Municipality has participated. Furthermore, is was important to have a patient perspective on the solution and therefore a group discussion with cardiac patients has been held. To analyze cost and quality a cost-effectiveness analysis has been carried out.  

\textbf{Findings} \newline
The core findings in this project is with regard to the interviews that has been performed and the economy analysis that has been made. By the interviews it can be concluded that both healthcare professionals and cardiac patients are happy and satisfied with the way cardiac rehabilitation is being performed today. To replace center-based rehabilitation will not be a seen as the best solution but to expand with telemedicine could possibly have a positive outcome. Due to the economy part of the project is has been evaluated that 488.383,33DKK will be saved per gained Quality Adjusted Life Year when exchanging center-based rehabilitation with telemedicine rehabilitation.  

\textbf{Conclusion} \newline
The outcome of this project has proven that telemedicine rehabilitation could be an effective and favorable expansion of the rehabilitation program that is being used today in Herning Municipality. With regard to a patient perspective it has been showed that telemedicine rehabilitation could result in a continuous maintenance of healthy lifestyle. Looking at the Danish Healthcare System it can be concluded that both the effectiveness of healthcare services and the financial part could benefit from the solution. When implementing a new healthcare technology, it is important to keep in mind that the legislation plays an important role and that the solution has to be evidence based to be an integrated part of the rehabilitation program.    






















