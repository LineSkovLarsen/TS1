\chapter{Conclusion}

The aim of this project was to elaborate on how an ICT solution for cardiac patients can effect both the Danish Healthcare System as well as cardiac patients. Furthermore, the purpose was to investigate what challenges and barriers such solution encounter when deployed in Denmark and other countries. 

To investigate this area, multiple methodologies have been applied. Primarily interviews have been carried out to investigate the opinion of both healthcare professionals and cardiac patient, who would be effected by the implementation of the ICT solution. 

From the literature research and the conducted interviews, relevant findings regarding this area has been shown. Cardiac patients and healthcare professionals who were interviewed in this project are doubtful that an ICT solution is able to replace the centre-based rehabilitation but they are convinced that a combination of both telemedicine and standard CR could improve quality of life. Both healthcare professionals and cardiac patients implies that an ICT solution at home could motivate the patients to keep up the healthy lifestyle and exercise training after the 12-week rehabilitation program at the centre. On the other hand, studies shows that a complete transition from standard cardiac rehabilitation to telemedicine solution will improve the quality of adjusted life years and the patients self-management. 

To analyse the financial and effectiveness impact of implementing an ICT solution as vCare in Herning Municipality the Cost-Effectiveness Analysis was chosen. To examine this the Incremental Cost-Effectiveness Ratio was calculated. To do so, a control group and intervention group is needed to compare the existing method with the intervention technology. The control group was the centre-based CR treatment in Herning Municipality. For the intervention group three different technology setup was obtained and the Silver package was chosen as the comparable setup with the control group.

The analysis indicated that 488.383,33DKK would be saved per gained QALY when exchanging standard CR with telemedicine. Basically this suggest that QALY is increased and the cost is declining when introducing an ICT solution in cardiac rehabilitation. The data for the analysis was conducted in collaboration with Herning Municipality, comparable studies and interviews with healthcare professionals.

Re hospitalization was included in the Cost-Effectiveness Analysis as several cardiac patients is readmitted after being discharged. A study conducted in Belgium investigated this matter and showed that the readmission rate decreased from 0.37 to 0.17. In Herning Municipality the readmission rate is approximately 31\% for standard cardiac rehabilitation and it is estimated to decrease to 17\% with the implementation of telemedicine. By this analysis it is seen that ICT might help to prevent readmission of cardiac patients. 
\newpage
When implementing telemedicine solutions in Denmark it is important to consider the structure of the Danish Healthcare System. The municipalities in Denmark are in charge of the rehabilitation of patients and they are the ones to choose the acknowledged method of rehabilitation. The municipalities are to follow the European and Danish legislation which include the regulation on public tendering. 

The research throughout this project indicates that an ICT solution for cardiac patients would have a positive effect on the Healthcare System through lower cost and reduced readmissions. The danish cardiac patients are sceptical, but studies shows that patients will gain quality of life when replacing standard rehabilitation with telemedicine rehabilitation.  



