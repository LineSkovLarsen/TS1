\chapter{Konklusion}

%I konklusionen gives en samlet konklusion på projektarbejdet og procesarbejdet. Hvad er lykkedes, hvad er ikke lykkedes og hvad er årsagen til dette. Konklusionen skal indeholde et klart budskab og forholde sig objektivt til de krav, der er opstillet i projektet og de re- sultater som I har opnået. Det er vigtigt at konklusionen hænger sammen med problem- formuleringen og den skal give svar på de spørgsmål, som er opstillet for projektet. Der kan med fordel konkluderes på MOSCOW-beskrivelsen i kravspecifikationen, hvis denne form for beskrivelse har været anvendt.
%Afsnittet skal desuden sammenfatte de slutninger, der kan drages af de resultater, som er omtalt i rapportens tidligere afsnit. Disse slutninger kan være såvel positive som negative. I skal ikke undlade at undertrykke de negative fund – har en metode f. eks. vist sig uegnet, bør det opfattes og fremhæves som et bidrag til ens erfaringsmateriale, og ikke skjules som et personligt nederlag.
%I konklusionen trækkes desuden de store linjer op. Væsentlige kvantitative resultater kan nævnes, hvorimod den detaljerede redegørelse og diskussion henvises til det foregående afsnit ”Diskussion af resultater”.
%Endelig skal der konkluderes på de vigtigste erfaringer fra selve processen.
%Som helhed skal konklusionerne, både den samlede konklusion og de individuelle kon- klusioner, være objektive og baseret på facts.
%Som helhed skal konklusionen være objektiv og baseret på facts.

Litteraturen har vist, at der er behov for en ny procedure, hvormed højrisiko patienter med for små biologiske hjerteklapper, kan undgå reoperation ved åben hjertekirurgi. Den nye procedure indenfor interventionel kardiologi, BF-ViV, har vist sig at være en lovende metode til denne patientgruppe. 

Under proceduren bliver fraktureringen på nuværende tidspunkt observeret af operatøren. På baggrund af de få og usikre identifikationer, som operatøren får ved fraktureringsøjeblikket, er det blevet påvist, at der er behov et for system, som kan støtte og bekræfte operatøren i, at fraktureringen har fundet sted. 

I udviklingsprocessen er der blevet identificeret en tryktransducer, som kan måle de nødvendige tryk, der kræves for at frakturere hjerteklappen. Denne tryktransducer er anvendelige til udvikling af prototypen og er blevet anvendt til forsøg in vitro. Det har dog ikke været muligt at finde et tryktransducer, som kan anvendes i klinisk miljø og opfylder de tekniske egenskaber. 

Til at detektere en frakturering er der blevet udviklet en detekteringsalgoritme. Det er blevet observeret, hvorvidt algoritmen opfanger en frakturering ved forskellige ballon- typer og størrelser. Det kan her konkluderes, at detekteringsalgoritmen ikke fungerer optimalt ved andre scenarier end det, hvorfra algoritmen er blevet udviklet. 

Det kan konkluderes, at der er blevet udviklet et detekteringssystem, som kan detektere en frakturering. Dette er dog under forudsætning af, at der anvendes den udviklede POM-stent, som efterligner Mitroflow 21 mm samt en 22 mm Atlas$^\circledR$ Gold ballon.    

På baggrund af accepttesten kan det konkluderes, at detekteringssystem kan benyttes ved brug af en True\texttrademark \ Dilatation ballon, 22 mm, under forudsætning af, at detekteringsalgoritmen ændres til threshold på 2. Ved disse parametre kan systemet altså benyttes til forskningsprojektet, Delstudie 2.  



