\chapter{Analysis}

\label{analysis}

\subsection{Relative’s Experiences of cardiac Patient’s telemedicine rehabilitation}
It is known that it can be stressful to be a relative to cardiac patients. Most often relatives help with home exercises, medicine dosage and transportation to and from the hospital. They participate in discussions about the patient’s illness and they do housekeeping and practical activities at home, which the patient isn't capable of doing. Research has shown that relatives are in risk of being a patient themselves as a consequence of the stressful job it is to take care of the patient \cite{4, 5}. Therefore, telemedicine rehabilitation is being offered to reduce relative’s homecare. By introducing telemedicine rehabilitation relatives feel more comfortable and secure as the patient is being monitored and healthcare staff react if the patient’s measurements are to be concerned about. By an interview of 13 cardiac patients who participated in telemedicine rehabilitation the results indicated that relatives find telemedicine equipment easy to use and the use of telemedicine motivates the patient to be more active in their own treatment \cite{12}.  

A research has taken place in Denmark where the patient did weekly blood pressure- and weight measurements. A heart rate monitor was used three times a week under physical conditions. Data were shown on an application via smartphone and hereby the patient, relatives and healthcare staff were able to follow the patient’s state of health. For the patients it was a relief that they were able to do exercises and health measurements at home and hereby they were able to do so according to work schedule as well as motivation and mental energy. Furthermore, less hospital visits removes focus on the disease and makes the patient feel more normal and less ill. Hereby patients experience higher quality of life as they feel healthier \cite{Bregendahl_2016}.  

Relatives experienced that everyday life were more normal by using telemedicine rehabilitation as they were able to continue everyday routines and spent less time taking care of the patient. They experienced more freedom as they didn’t have to take the patient to rehabilitation classes, regulate diet and take care of medicine. It indicates that relatives to patients using telemedicine rehabilitation gain more freedom and less concern and responsibility \cite{7}.  

\subsection{Impact of ICT in homecare}
Existing studies describing the use of ICT in homecare are predominated by positive responses from both chronically ill patients and healthcare professional. As an example, healthcare professional’s opinion is that their work has been facilitated by introducing ICT in homecare. Most studies show that communication between patients and healthcare professional was improved by using ICT. Furthermore, the use of ICT showed cost savings. However, it is important to keep in mind that the use of ICT cannot replace face to face consultations but is an ideal complement \cite{ICT}. It is important to keep in mind that telemedicine supporting already integrated care is associated with the development of new roles within the healthcare system. Ideally, new structures of care delivery at an operational level needs to be supported by corresponding changes at institutional level \cite{countries}. Therefore, by introducing telemedicine both patients and healthcare professional has to be openminded to this new technology and adaptable to change already known working methods. Hence, the development of ICT in homecare should be seen as a learning process and will constantly be evolving and improved based on the ongoing use. 

Another important impact of ICT is the information flow between healthcare professionals. Effective interprofessional communication is highly important within the healthcare system but is seen to be critical when teams are not co-located. For this reason, healthcare professionals who has been in use of an ICT solution pointed out how information via ICT potentially could have and positive effect on patient care and collaboration \cite{barrier}.

Furthermore, exchanging information with patients, follow up and motive them to keep working out and keep having a healthy lifestyle is seen to be easier with ICT. Patients are able to log information, send documents and ask questions more frequently which increases the communication and contact between patient and healthcare professionals. Having more regular discussions with the patient will facilitate more comprehensive and effective collaboration to the patient \cite{barrier}.   

\subsection{Comparison with telemedicine solution for COPD patients}

\textbf{Remember to wrote that it was a school project} \newline
Telemedicine solutions have been tested in pilot projects in different cities in Denmark. The projects have shown that telemedicine can provide financial benefits as well as better and more consistent patients progress and more self-reliant patients \cite{KOL_1}. In 2016 the government, \textit{Kommunernes Landsforening} and Danish Regions did an agreement to offer telemedicine home monitoring to citizens with Chronic obstructive pulmonary disease (COPD) throughout the country by the end of 2019 \cite{KOL_2}. 

In 2014 a pilot project took place in the municipality of Skanderborg were 15 COPD patients were included. After participating in the project, the patients were interviewed to give their perspective on the telemedicine solution. Overall the patients were very satisfied for the solution and especially as they had the freedom to do measurements and exercises whenever they wanted and did not depend on a specific time schedule at the hospital. The only disadvantage the patients were aware of was the connection which sometimes was a bit unstable. For the patients it was very important that picture and sound on the platform was clear and was working optimal at all time, otherwise they lost the motivation. An important observation at this interview was how the patients experiences the social aspect. The patients were used to do exercises at the gym in classes with other patients. Now they had to do exercise at home where they were able to see and talk to each other through the screen. One of the patient’s mentioned that it was a good solution but only for a short time. To him the social aspect was very important, and he didn’t experience the social interaction the same way as he did at the gym. Another important observation was one of the patients who was too ill to get to the hospital and therefore he wasn’t capable of participate at the exercise classes. But by this telemedicine solution he was able to do exercise at home and in the end of the project his physical condition was so good that he was able to do his normal routines at home and also to leave home and go to the hospital. Therefore, this telemedicine solution definitely was an important help to make him feel and get better, Appendix 1. 


\subsection{Challenges within telemedicine rehabilitation} 
The telemedicine solution collides with the GPs’ individual approach, where knowledge on patients’ reaction patterns and personal relationship to the patient is important when assessing the patient and deciding the right intervention. By the use of telemedicine, the GPs’ are not able to look at the patient’s overall condition and use knowledge about the patient’s normal reaction. By using telemedicine GPs’ will be looking at measurements measured by the patients themselves and that won’t give the same overall understanding on the patients’ physical condition \cite{Emergence}.  

Furthermore, communication through ICT is seen to be more impersonal and to build up trust to the patient is much more difficult compared to face to face meetings. Visual information such as body language, person interaction and empathy are very important for the therapeutic relationship and this is seen to be a barrier to the effective collaboration between healthcare professional and the patient \cite{barrier}.  

There are certain technological skills necessary for operating ICT. The majority of cardiac patients are older adults and may not be familiar or comfortable using ICT. Some patients might not be used to use technology on a daily basis and therefore the ICT solution can be a difficult solution for that specific patient group. Additionally, some patients might live in rural areas where adequate internet access is not available. This is seen to be a barrier which has to be considered when introducing ICT \cite{barrier}. 
   

\subsection{Effects and barriers in implementation of telemedicine solutions for chronic patients across 3 European countries}
Deployment of ICT solutions can be tough. Some barriers is "workforce preparation and organizational aspects; regulatory and ethical issues; business model and reimbursement modalities; and, technological
factors. During the deployment process, appropriateness of the organizational dimension of the project constitutes, with no doubt, a major priority to ensure positive outcomes"\cite{effects}.
Throughout their studies in Barcelona, Athens and Trondheim they found two factors that must be satisfied when deploying ICT in healthcare. Firstly the intervention of ICT should take the patients characteristics in to account. Secondly the businessmodel, in particular the reimbursement modalities, incentives and shared risk among the involved people in the process\cite{effects}.
%Note til overstående at i DK er betalings formen anderledes, men der skal også tages højde for reimbursement. Det er i særdeleshed her DK adskiller sig fra de andre lande.





What impact would an ICT solution for rehabilitation have on both cardiovascular patients and the Danish Healthcare System?
%Brug her evt denne artikel til at beskrive det danskesygehusvæsens styrker og svagheder. 
\cite{Healthconsensus}.
%lovgivning validitet i forhold til den kendte CR behandling 


How can ICT be used to shorten hospital stay for cardiovascular patients?
% Se på statistikker
% Hvor lange indlæggelser/genindlæggelser har patienter
% Kan vi se noget ud fra tallene på KOL projektet.
%Hvordan vil man beregne cost effectivnes - ses i artikel om kommende projekt 
% anvende fagfolk til at tilkendegive deres mening 

Which barriers/challenges can such system meet in implementation?

%Mangel på procedure? - fælles medicinsk udbud
%Her skal vi se på opbygningen af det danske sundhedsvæsen og betaling metoden. 
%Patientrettigheder side 45 \cite{DKhealthreview}
%Noget med udbud på it systemer over en hvis pris?
% continually training of healthcare professionals
% legal, limitations and ethics \cite{considerations}
%Data sikkerhed ny data forordring
%oplæring af patienter


%Nedstående skal undersøges
"Studies have shown that telemental health has proven to reduce both
healthcare costs and patient costs. Break-even analysis of multiple studies, however, showed that in order
to be cost-effective, a certain number of consultations must take place per year to justify the capital
investment costs of implementing such programs (Hilty, et al., 2013)". \cite{considerations}

%Nedstående kan anvendes som argumentation
 %"These programs are only successful, though, if patients take initiative in the self-management of their chronic diseases. Furthermore, once equipment is installed, extensive patient or caregiver education is required in order to properly use these devices and obtain accurate measurements. Once the data is transmitted daily to the control center in the hospital, a nurse reviews the data for any alerts, notifying the patient’s provider for follow-up if these results are not consistent with that patient’s usual state of health. The benefits of home health telemedicine, or telehealth, include improved compliance, improved patient and caregiver satisfaction, decreased anxiety, increased quality of life, and patient empowerment through active participation in managing their own diseases. From a healthcare professional perspective, telehealth offers daily monitoring and management, preventative care, early intervention, improved provider-patient communication. Furthermore, it creates efficiencies and cost savings by allowing providers to manage and monitor multiple patients simultaneously, which reduces ER visits and improves allocation of scarce medical resources." \cite{considerations}.
 
 % Nedstående er mulige ting der kan anvendes forskellige steder i rapporten
 %"The patients' self-monitoring of the disease should be enhanced and technologies for self-monitoring should be evaluated and the quality of the monitoring should be assured."\cite{NationalBoardofHealth}


\section{Cost-effective}
%HEr skal være en tabel per cost hvor vi går mere i detaljer med priserne
%Vi skal forklare hvor detaljeret udetaljeret modellen er 
%Hvor vi har data fra og hvordan de skal læses
%Lav modeller der visualisere 
%vis udregningre

\chapter{Discussion}



%DEt er vigtigt at vi kommer ind på etik, det her med at patienten skal tilbydes bedst mulige behandling. Hvis de siger nej til fase 2 i samarbejde med kpmmunen vil de måske sige ja til VCare da det foregår i hjemmet --> patientens autonomi

\chapter{Recommendation}

\chapter{Perspectivation}
%Kan også være refleksion
\subsection{Ongoing project in Netherlands}

A project in Netherlands is looking into the impact of ICT solution for a group of cardiac patients. The patient group has been diagnosed with coronary artery disease (CAD). The result of the trial it not yet submitted as the trial is ongoing. The study is looking into 300 patients whereas 150 are restricted as a control group. The control group will receive normal Cardiac Rehabilitation (CR) treatment and the intervention group will receive home-based telemedicine rehabilitation\cite{CAD}. 
The result of this study could have a great impact on this project.
