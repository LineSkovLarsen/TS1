\chapter{Analysis and discussion}

\subsection{Relative’s Experiences of cardiac Patient’s telemedicine rehabilitation}
It is known that it can be stressful to be a relative to cardiac patients. Most often relatives help with home exercises, medicine dosage and transportation to and from the hospital. They participate in discussions about the patient’s illness and they do housekeeping and practical activities at home, which the patient isn't capable of doing. Research has shown that relatives are in risk of being a patient themselves as a consequence of the stressful job it is to take care of the patient \cite{4} \cite{5}. Therefore, telemedicine rehabilitation is being offered to reduce relative’s homecare. By introducing telemedicine rehabilitation relatives feel more comfortable and secure as the patient is being monitored and healthcare staff react if the patient’s measurements are to be concerned about. By an interview of 13 cardiac patients who participated in telemedicine rehabilitation the results indicated that relatives find telemedicine equipment easy to use and the use of telemedicine motivates the patient to be more active in their own treatment \cite{12}.  

A research has taken place in Denmark where the patient did weekly blood pressure- and weight measurements. A heart rate monitor was used three times a week under physical conditions. Data were shown on an application via smartphone and hereby the patient, relatives and healthcare staff were able to follow the patient’s state of health. For the patients it was a relief that they were able to do exercises and health measurements at home and hereby they were able to do so according to work schedule as well as motivation and mental energy. Furthermore, less hospital visits removes focus on the disease and makes the patient feel more normal and less ill. Hereby patients experience higher quality of life as they feel healthier \cite{Bregendahl_2016}.  

Relatives experienced that everyday life were more normal by using telemedicine rehabilitation as they were able to continue everyday routines and spent less time taking care of the patient. They experienced more freedom as they didn’t have to take the patient to rehabilitation classes, regulate diet and take care of medicine. It indicates that relatives to patients using telemedicine rehabilitation gain more freedom and less concern and responsibility \cite{7}.  
