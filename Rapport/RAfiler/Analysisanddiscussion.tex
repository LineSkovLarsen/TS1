\chapter{Analysis and discussion}

\subsection{Relative’s Experiences of cardiac Patient’s telemedicine rehabilitation}
It is known that it can be stressful to be a relative to cardiac patients. Most often relatives help with home exercises, medicine dosage and transportation to and from the hospital. They participate in discussions about the patient’s illness and they do housekeeping and practical activities at home, which the patient isn't capable of doing. Research has shown that relatives are in risk of being a patient themselves as a consequence of the stressful job it is to take care of the patient \cite{4, 5}. Therefore, telemedicine rehabilitation is being offered to reduce relative’s homecare. By introducing telemedicine rehabilitation relatives feel more comfortable and secure as the patient is being monitored and healthcare staff react if the patient’s measurements are to be concerned about. By an interview of 13 cardiac patients who participated in telemedicine rehabilitation the results indicated that relatives find telemedicine equipment easy to use and the use of telemedicine motivates the patient to be more active in their own treatment \cite{12}.  

A research has taken place in Denmark where the patient did weekly blood pressure- and weight measurements. A heart rate monitor was used three times a week under physical conditions. Data were shown on an application via smartphone and hereby the patient, relatives and healthcare staff were able to follow the patient’s state of health. For the patients it was a relief that they were able to do exercises and health measurements at home and hereby they were able to do so according to work schedule as well as motivation and mental energy. Furthermore, less hospital visits removes focus on the disease and makes the patient feel more normal and less ill. Hereby patients experience higher quality of life as they feel healthier \cite{Bregendahl_2016}.  

Relatives experienced that everyday life were more normal by using telemedicine rehabilitation as they were able to continue everyday routines and spent less time taking care of the patient. They experienced more freedom as they didn’t have to take the patient to rehabilitation classes, regulate diet and take care of medicine. It indicates that relatives to patients using telemedicine rehabilitation gain more freedom and less concern and responsibility \cite{7}.  

\subsection{Comparison with telemedicine solution for COPD patients}
Telemedicine solutions have been tested in pilot projects in different cities in Denmark. The projects have shown that telemedicine can provide financial benefits as well as better and more consistent patients progress and more self-reliant patients \cite{KOL_1}. In 2016 the government, \textit{Kommunernes Landsforening} and Danish Regions did an agreement to offer telemedicine home monitoring to citizens with Chronic obstructive pulmonary disease (COPD) throughout the country by the end of 2019 \cite{KOL_2}. 

In 2014 a pilot project took place in the municipality of Skanderborg were 15 COPD patients were included. After participating in the project, the patients were interviewed to give their perspective on the telemedicine solution. Overall the patients were very satisfied for the solution and especially as they had the freedom to do measurements and exercises whenever they wanted and did not depend on a specific time schedule at the hospital. The only disadvantage the patients were aware of was the connection which sometimes was a bit unstable. For the patients it was very important that picture and sound on the platform was clear and was working optimal at all time, otherwise they lost the motivation. An important observation at this interview was how the patients experiences the social aspect. The patients were used to do exercises at the gym in classes with other patients. Now they had to do exercise at home where they were able to see and talk to each other through the screen. One of the patient’s mentioned that it was a good solution but only for a short time. To him the social aspect was very important, and he didn’t experience the social interaction the same way as he did at the gym. Another important observation was one of the patients who was too ill to get to the hospital and therefore he wasn’t capable of participate at the exercise classes. But by this telemedicine solution he was able to do exercise at home and in the end of the project his physical condition was so good that he was able to do his normal routines at home and also to leave home and go to the hospital. Therefore, this telemedicine solution definitely was an important help to make him feel and get better, Appendix 1. 

\subsection{Challenges within telemedicine rehabilitation} 
The telemedicine solution collides with the GPs’ individual approach, where knowledge on patients’ reaction patterns and personal relationship to the patient is important when assessing the patient and deciding the right intervention. With the use of telemedicine, the GPs’ are not able to look at the patient’s overall condition and use knowledge about the patient’s normal reaction. By using telemedicine GPs’ will be looking at measurements measured by the patients themselves and that won’t give the same overall understanding on the patients’ physical condition \cite{Emergence}.  


What impact would an ICT solution for rehabilitation have on both cardiovascular patients and the Danish Healthcare System?
%Brug her evt denne artikel til at beskrive det danskesygehusvæsens styrker og svagheder. 
\cite{Healthconsensus}.


How can ICT be used to shorten hospital stay for cardiovascular patients?
% Se på statistikker
% Hvor lange indlæggelser/genindlæggelser har patienter
% Kan vi se noget ud fra tallene på KOL projektet.
% sammenligne teknologier
% anvende fagfolk til at tilkendegive deres mening 

Which barriers/challenges can such system meet in implementation?

% Teknologi forskrækket
%Mangel på procedure? - fælles medicinsk udbud
%Her skal vi se på opbygningen af det danske sundhedsvæsen og betaling metoden. 
%Patientrettigheder side 45 \cite{DKhealthreview}
